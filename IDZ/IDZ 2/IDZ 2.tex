\documentclass[a4paper,12pt]{article}
\usepackage{amsmath}
\usepackage{cmap}					% поиск в PDF
\usepackage{mathtext} 				% русские буквы в формулах
\usepackage[T2A]{fontenc}			% кодировка
\usepackage[utf8]{inputenc}			% кодировка исходного текста
\usepackage[english,russian]{babel}	% локализация и переносы
\usepackage{amssymb}				    % Для красивого (!) \mathbb с  буквами и цифрами
\usepackage{mathbbol}

% Изменим формат \section и \subsection:
\usepackage{titlesec}
\titleformat{\section}
{\vspace{1cm}\centering\LARGE\bfseries}	% Стиль заголовка
{}										% префикс
{0pt}									% Расстояние между префиксом и заголовком
{} 										% Как отображается префикс
\titleformat{\subsection}				% Аналогично для \subsection
{\Large\bfseries}
{}
{0pt}
{}

%% Отступы между абзацами и в начале абзаца 
\setlength{\parindent}{0pt}
\setlength{\parskip}{\medskipamount}
\begin{document}
	\section{Индивидульное домашнее задание 2 \\ Шумилкин Андрей. Группа 163 \\ Вариант 43. } 
	\subsection{Задача 1} 
	Пусть
	 \[
	 A=
	\begin{pmatrix}
	1 & 2 & 3 & 4 & 5 & 6 & 7 & 8 \\
	4 & 5 & 2 & 6 & 8 & 3 & 7 & 1 \\
	\end{pmatrix} = (4632581)(7)
	\]
	
	 \[
	 B=
	 \begin{pmatrix}
	 1 & 2 & 3 & 4 & 5 & 6 & 7 & 8 \\
	 5 & 8 & 1 & 3 & 2 & 7 & 6 & 4 \\
	 \end{pmatrix} = (528431)(76)
	 \]
	 
	 \[
	 C=
	 \begin{pmatrix}
	 1 & 2 & 3 & 4 & 5 & 6 & 7 & 8 \\
	 4 & 3 & 6 & 7 & 2 & 8 & 5 & 1 \\
	 \end{pmatrix}
	 \]
	 
	 Тогда(т.к. lcm(7,1) = 7): $A^7 = id = A^{14} \Rightarrow A^{19} = A^5$
	  \[
	  A^{19}=
	  \begin{pmatrix}
	  1 & 2 & 3 & 4 & 5 & 6 & 7 & 8 \\
	  5 & 6 & 4 & 8 & 3 & 1 & 7 & 2 \\
	  \end{pmatrix}
	  \]
	 
	  \[
	  B^{-1}=
	  \begin{pmatrix}
	  1 & 2 & 3 & 4 & 5 & 6 & 7 & 8 \\
	  3 & 5 & 4 & 8 & 1 & 7 & 6 & 2 \\
	  \end{pmatrix}
	  \]
	  
	  $D = A^{19} \cdot B^{-1}$
	  
	  \[
	  D=
	  \begin{pmatrix}
	  1 & 2 & 3 & 4 & 5 & 6 & 7 & 8 \\
	  4 & 3 & 8 & 2 & 5 & 7 & 1 & 6 \\
	  \end{pmatrix} = (4238671)(5)
	  \]
	  
	  Тогда $D^{7} = id = D^{140} \Rightarrow D^{145} = D^{5}$.
	  
	   \[
	   D^{145}=
	  \begin{pmatrix}
	  1 & 2 & 3 & 4 & 5 & 6 & 7 & 8 \\
	  6 & 1 & 4 & 7 & 5 & 3 & 8 & 2 \\
	  \end{pmatrix} = F
	  \]
	   
	   $F \cdot X = C$\\
	   $F^{-1} \cdot F \cdot X = F^{-1} \cdot C$\\
	   $X = F^{-1} \cdot C$\\
	   
	   \[
	   F^{-1}=
	   \begin{pmatrix}
	   1 & 2 & 3 & 4 & 5 & 6 & 7 & 8 \\
	   2 & 8 & 6 & 3 & 5 & 1 & 4 & 7 \\
	   \end{pmatrix}
	   \]
	   
	   \[
	   F^{-1} \cdot C=
	   \begin{pmatrix}
	   1 & 2 & 3 & 4 & 5 & 6 & 7 & 8 \\
	   3 & 6 & 1 & 4 & 8 & 7 & 5 & 2 \\
	   \end{pmatrix} = X
	   \]
	   
	   \[
	   Ответ: \ X=
	   \begin{pmatrix}
	   1 & 2 & 3 & 4 & 5 & 6 & 7 & 8 \\
	   3 & 6 & 1 & 4 & 8 & 7 & 5 & 2 \\
	   \end{pmatrix}
	   \]
	   
	   \subsection{Задача 2} 
	   Будем смотреть сколько чисел меньше рассматриваемого числа находится правее него. Сумма этих значений будет равна числу инверсий и с помощью его четности мы и определим четность подстановки.
	   Видно, что для любого $\sigma(i)$ из $i = 1 \dots 187$ все $\sigma(j) > \sigma(i), j = (i + 1) \dots 188$ и все $\sigma(t) < \sigma(i), t=189 \dots 235$. Последнее условие также выполняется для i = 188.\\
	   Значит каждый элемент $i = 1 \dots 188$ дает (235 - 189 + 1 = 47) инверсий.\\
	   188(четн.) $\cdot$ 47(нечетн.) = четное число. \\
	   Ответ: Подстановка четна.
	   
	   \subsection{Задача 3}
	   Раскроем определитель по первому столбцу:
	   \[
	   D = 3 \cdot
	   \begin{vmatrix}
	   0 & 0 & 3 & 3 & 0 \\
	   5 & 3 & 0 & 1 & 1 \\
	   4 & 3 & 3 & 1 & 0 \\
	   3 & 6 & 3 & 4 & 0 \\
	   7 & 5 & 4 & 0 & 4 \\
	   \end{vmatrix} 
	   -
	   \begin{vmatrix}
	   0 & 0 & 3 & 3 & 0 \\
	   5 & 3 & 0 & 1 & 1 \\
	   1 & 0 & 4 & 0 & 2 \\
	   4 & 3 & 3 & 1 & 0 \\
	   3 & 6 & 3 & 4 & 0 \\
	   \end{vmatrix}
	   \]
	    
	    Раскроем оба этих определителя по последнему столбцу:
	    \[
	    D = 
	    -3 \cdot
	    \begin{vmatrix}
	    0 & 0 & 3 & 3 \\
	    4 & 3 & 3 & 1 \\
	    3 & 6 & 3 & 4 \\
	    7 & 5 & 4 & 0 \\
	    \end{vmatrix} 
	    + 12 \cdot 
	    \begin{vmatrix}
	    0 & 0 & 3 & 3 \\
	    5 & 3 & 0 & 1 \\
	    4 & 3 & 3 & 1 \\
	    3 & 6 & 3 & 4 \\
	    \end{vmatrix}
	    +
	    \begin{vmatrix}
	    0 & 0 & 3 & 3 \\
	    1 & 0 & 4 & 0 \\
	    4 & 3 & 3 & 1 \\
	    3 & 6 & 3 & 4 \\
	    \end{vmatrix}
	    - 2 \cdot
	    \begin{vmatrix}
	    0 & 0 & 3 & 3 \\
	    5 & 3 & 0 & 1 \\
	    4 & 3 & 3 & 1 \\
	    3 & 6 & 3 & 4 \\
	    \end{vmatrix}
	    \]
	    Заметим, что второй определитель равен четвертому. Тогда:
	    \[
	    D =
	    -3 \cdot
	    \begin{vmatrix}
	    0 & 0 & 3 & 3 \\
	    4 & 3 & 3 & 1 \\
	    3 & 6 & 3 & 4 \\
	    7 & 5 & 4 & 0 \\
	    \end{vmatrix} 
	    + 10 \cdot 
	    \begin{vmatrix}
	    0 & 0 & 3 & 3 \\
	    5 & 3 & 0 & 1 \\
	    4 & 3 & 3 & 1 \\
	    3 & 6 & 3 & 4 \\
	    \end{vmatrix}
	    + 
	    \begin{vmatrix}
	    0 & 0 & 3 & 3 \\
	    1 & 0 & 4 & 0 \\
	    4 & 3 & 3 & 1 \\
	    3 & 6 & 3 & 4 \\
	    \end{vmatrix}
	    \]
	    В каждом из определителей отнимем последний столбец от предпоследнего. Эта операция не меняет определитель.
	    \[
	    D =
	    -3 \cdot
	    \begin{vmatrix}
	    0 & 0 & 0 & 3 \\
	    4 & 3 & 2 & 1 \\
	    3 & 6 & -1 & 4 \\
	    7 & 5 & 4 & 0 \\
	    \end{vmatrix} 
	    + 10 \cdot 
	    \begin{vmatrix}
	    0 & 0 & 0 & 3 \\
	    5 & 3 & -1 & 1 \\
	    4 & 3 & 2 & 1 \\
	    3 & 6 & -1 & 4 \\
	    \end{vmatrix}
	    +
	    \begin{vmatrix}
	    0 & 0 & 0 & 3 \\
	    1 & 0 & 4 & 0 \\
	    4 & 3 & 2 & 1 \\
	    3 & 6 & -1 & 4 \\
	    \end{vmatrix}
	    \]
	    Раскроем каждый определитель по первой строке:
	    \[
	    D =
	    9 \cdot
	    \begin{vmatrix}
	    4 & 3 & 2 \\
	    3 & 6 & -1 \\
	    7 & 5 & 4 \\
	    \end{vmatrix} 
	    - 30 \cdot 
	    \begin{vmatrix}
	    5 & 3 & -1 \\
	    4 & 3 & 2 \\
	    3 & 6 & -1 \\
	    \end{vmatrix}
	    -3 \cdot
	    \begin{vmatrix}
	    1 & 0 & 4 \\
	    4 & 3 & 2 \\
	    3 & 6 & -1 \\
	    \end{vmatrix}
	    \]
	    В первом определителе прибавим последний столбец к первому с коэффициентом (3) и ко второму с коэффициентом (6).\\
	    Во втором прибавим первую строку ко второй и третьей с коэффициентами (-1) и (-2) соответственно.\\
	    В третьем прибавим вторую строку к третьей с коэффициентом (-2). 
	    \[
	    D =
	    9 \cdot
	    \begin{vmatrix}
	    10 & 15 & 2 \\
	    0 & 0 & -1 \\
	    19 & 29 & 4 \\
	    \end{vmatrix} 
	    - 30 \cdot 
	    \begin{vmatrix}
	    5 & 3 & -1 \\
	    -1 & 0 & 3 \\
	    -7 & 0 & 1 \\
	    \end{vmatrix}
	    -3 \cdot
	    \begin{vmatrix}
	    1 & 0 & 4 \\
	    4 & 3 & 2 \\
	    -5 & 0 & -5 \\
	    \end{vmatrix}
	    \]
	    Раскроем первый определитель по последнему столбцу.\\
	    Второй -- по второму столбцу.\\
	    Третий -- по второму столбцу.
	    \[
	    D =
	    9 \cdot
	    \begin{vmatrix}
	    10 & 15 \\
	    19 & 29 \\
	    \end{vmatrix} 
	    + 90 \cdot 
	    \begin{vmatrix}
	    -1 & 3 \\
	    -7 & 1 \\
	    \end{vmatrix}
	    - 9 \cdot
	    \begin{vmatrix}
	    1 & 4 \\
	    -5 & -5 \\
	    \end{vmatrix}
	    \]
	    \[
	    D =
	    9 \cdot 5
	    + 90 \cdot 20
	    - 9 \cdot 15 = 1710
	    \]
	    Ответ: 1710.
	    
	    \subsection{Задача 4}
	    
	    \[
	    \begin{pmatrix}
	    7 & 2 & 9 & x & 5 & 7 & 2 \\
	    2 & 3 & x & 9 & 6 & 8 & 3 \\
	    9 & x & 3 & 2 & 10 & 1 & 7 \\
	    x & 9 & 2 & 7 & 4 & 6 & x \\
	    5 & 6 & 10 & 4 & 2 & x & 10 \\
	    7 & 8 & 1 & 6 & x & 7 & 2 \\
	    2 & 3 & 7 & x & 10 & 2 & 3 \\
	    \end{pmatrix}
	    \]
	    
	    Добавим к 1-ой строке 7-ую, домноженную на -1:
	    
	    \[
	    \begin{pmatrix}
	    5 & -1 & 2 & 0 & -5 & 5 & -1 \\
	    2 & 3 & x & 9 & 6 & 8 & 3 \\
	    9 & x & 3 & 2 & 10 & 1 & 7 \\
	    x & 9 & 2 & 7 & 4 & 6 & x \\
	    5 & 6 & 10 & 4 & 2 & x & 10 \\
	    7 & 8 & 1 & 6 & x & 7 & 2 \\
	    2 & 3 & 7 & x & 10 & 2 & 3 \\
	    \end{pmatrix}
	    \]
	    
	    Добавим к 1-ому столбцу 7-ый, домноженный на -1:
	    
	    \[
	    \begin{pmatrix}
	    6 & -1 & 2 & 0 & -5 & 5 & -1 \\
	    -1 & 3 & x & 9 & 6 & 8 & 3 \\
	    2 & x & 3 & 2 & 10 & 1 & 7 \\
	    0 & 9 & 2 & 7 & 4 & 6 & x \\
	    -5 & 6 & 10 & 4 & 2 & x & 10 \\
	    5 & 8 & 1 & 6 & x & 7 & 2 \\
	    -1 & 3 & 7 & x & 10 & 2 & 3 \\
	    \end{pmatrix}
	    \]  
	    
	    Поменяем местами вторую и четвертую, а затем третью и пятую строчки. Далее поменяем третью с шестой и пятую и с седьмой. Т.к. у нас 4 обмена местами, то определитель не изменится($(-1)^4 = 1$).
	    
	    \[ D = 
	    \begin{pmatrix}
	    6 & -1 & 2 & 0 & -5 & 5 & -1 \\
	    0 & 9 & 2 & 7 & 4 & 6 & x \\
	    -5 & 6 & 10 & 4 & 2 & x & 10 \\
	    5 & 8 & 1 & 6 & x & 7 & 2 \\
	    -1 & 3 & 7 & x & 10 & 2 & 3 \\
	    -1 & 3 & x & 9 & 6 & 8 & 3 \\
	    2 & x & 3 & 2 & 10 & 1 & 7 \\
	    \end{pmatrix}
	    \]  
	    
	    Раскроем определитель по первому столбцу: 
		\[ 
		D = 6 \cdot
		\begin{pmatrix}
		9 & 2 & 7 & 4 & 6 & x \\
		6 & 10 & 4 & 2 & x & 10 \\
		8 & 1 & 6 & x & 7 & 2 \\
		3 & 7 & x & 10 & 2 & 3 \\
		3 & x & 9 & 6 & 8 & 3 \\
		x & 3 & 2 & 10 & 1 & 7 \\
		\end{pmatrix} 
		-5 \cdot 
		\begin{pmatrix}
		-1 & 2 & 0 & -5 & 5 & -1 \\
		9 & 2 & 7 & 4 & 6 & x \\
		8 & 1 & 6 & x & 7 & 2 \\
		3 & 7 & x & 10 & 2 & 3 \\
		3 & x & 9 & 6 & 8 & 3 \\
		x & 3 & 2 & 10 & 1 & 7 \\
		\end{pmatrix}
		-5 \cdot 
		\begin{pmatrix}
		-1 & 2 & 0 & -5 & 5 & -1 \\
		9 & 2 & 7 & 4 & 6 & x \\
		6 & 10 & 4 & 2 & x & 10 \\
		3 & 7 & x & 10 & 2 & 3 \\
		3 & x & 9 & 6 & 8 & 3 \\
		x & 3 & 2 & 10 & 1 & 7 \\
		\end{pmatrix}
		\]
		\[
		-
		\begin{pmatrix}
		-1 & 2 & 0 & -5 & 5 & -1 \\
		9 & 2 & 7 & 4 & 6 & x \\
		6 & 10 & 4 & 2 & x & 10 \\
		8 & 1 & 6 & x & 7 & 2 \\
		3 & x & 9 & 6 & 8 & 3 \\
		x & 3 & 2 & 10 & 1 & 7 \\
		\end{pmatrix}
		+ 
		\begin{pmatrix}
		-1 & 2 & 0 & -5 & 5 & -1 \\
		9 & 2 & 7 & 4 & 6 & x \\
		6 & 10 & 4 & 2 & x & 10 \\
		8 & 1 & 6 & x & 7 & 2 \\
		3 & 7 & x & 10 & 2 & 3 \\
		x & 3 & 2 & 10 & 1 & 7 \\
		\end{pmatrix}
		+ 2 \cdot 
		\begin{pmatrix}
		-1 & 2 & 0 & -5 & 5 & -1 \\
		9 & 2 & 7 & 4 & 6 & x \\
		6 & 10 & 4 & 2 & x & 10 \\
		8 & 1 & 6 & x & 7 & 2 \\
		3 & 7 & x & 10 & 2 & 3 \\
		3 & x & 9 & 6 & 8 & 3 \\
		\end{pmatrix}
		\]  
		
		В первый определитель вообще не даст $x^5$, так как все x на главной диагонали. То есть, если мы берем множитель $x^5$, то определению определителя(через перестановки) положение последнего элемента, который войдет в этот член определителя уже однозначно определено и это будет еще один $x$. Сразу отметим, что в остальных определителях положение последнего элемента входящего в член определителя так же определено(только там это уже некий коэффициент, а не $x$), что и позволит нам его легко посчитать. 
		Пронумеруем все определители, кроме первого, слева направо. \\
		Для определителя под номером $n$ назовем член определителя, котором содержится $x^5$ -- $m_n$.
	    Тогда:\\
	    $m_1 = 5x^5 \\ m_2 = -5x^5 \\ m_3 = 0 \\ m_4 = 2x^5 \\ m_5 = -x^5$.\\
	    Теперь посмотрим на число инверсий в перестановках, соответствующих данным членам. Перестановку для элемента $m_n$ обозначим $\sigma(m_n)$, а число инверсий в ней $N(\sigma(m_n))$.\\
	    \[
	    \sigma(m_1) = 
	    \begin{pmatrix}
	    1 & 2 & 3 & 4 & 5 & 6 \\
	    6 & 5 & 4 & 3 & 1 & 2 \\
	    \end{pmatrix}.\ N(\sigma(m_1)) = 5 + 4 + 3 + 2 = 14.
	    \]
	    \[
	    \sigma(m_2) = 
	    \begin{pmatrix}
	    1 & 2 & 3 & 4 & 5 & 6 \\
	    6 & 5 & 4 & 1 & 3 & 2 \\
	    \end{pmatrix}.\ N(\sigma(m_2)) = 5 + 4 + 3 + 1 = 13.
	    \]
	    \[
	    \sigma(m_3) = 
	    \begin{pmatrix}
	    1 & 2 & 3 & 4 & 5 & 6 \\
	    6 & 5 & 1 & 4 & 3 & 2 \\
	    \end{pmatrix}.\ N(\sigma(m_4)) = 5 + 4 + 2 + 1 = 12.
	    \]
	    \[
	    \sigma(m_4) = 
	    \begin{pmatrix}
	    1 & 2 & 3 & 4 & 5 & 6 \\
	    6 & 1 & 5 & 4 & 3 & 2 \\
	    \end{pmatrix}.\ N(\sigma(m_4)) = 5 + 3 + 2 + 1 = 11.
	    \]
	    \[
	    \sigma(m_5) = 
	    \begin{pmatrix}
	    1 & 2 & 3 & 4 & 5 & 6 \\
	    1 & 6 & 5 & 4 & 3 & 2 \\
	    \end{pmatrix}.\ N(\sigma(m_5)) = 4 + 3 + 2 + 1 = 10.
	    \]
	    
	    
	    Тогда, учитывая знаки под которыми входят данные члены в свои определители и вынеся их из них и домножив на коэффициенты, стоящие перед данными определителями в выражении определителя начальной матрицы, получаем для $x^5$, входящего в определитель начальной матрицы, следующее выражение:\\
	    $-5 \cdot (5x^5) -5 \cdot (5x^5) + (-2x^5) + 2 \cdot (-x^5) = -54x^5$.\\
	    
	    Ответ: -54.
	
	\end{document}