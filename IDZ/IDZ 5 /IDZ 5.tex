\documentclass[a4paper,12pt]{article}
\usepackage{amsmath}
\usepackage{cmap}					% поиск в PDF
\usepackage{mathtext} 				% русские буквы в формулах
\usepackage[T2A]{fontenc}			% кодировка
\usepackage[utf8]{inputenc}			% кодировка исходного текста
\usepackage[english,russian]{babel}	% локализация и переносы
\usepackage{amssymb}				    % Для красивого (!) \mathbb с  буквами и цифрами
\usepackage{mathbbol}

% Изменим формат \section и \subsection:
\usepackage{titlesec}
\titleformat{\section}
{\vspace{1cm}\centering\LARGE\bfseries}	% Стиль заголовка
{}										% префикс
{0pt}									% Расстояние между префиксом и заголовком
{} 										% Как отображается префикс
\titleformat{\subsection}				% Аналогично для \subsection
{\Large\bfseries}
{}
{0pt}
{}

%% Отступы между абзацами и в начале абзаца 
\setlength{\parindent}{0pt}
\setlength{\parskip}{\medskipamount}
\begin{document}
	\section{Индивидульное домашнее задание 5 \\ Шумилкин Андрей. Группа 165 \\ Вариант 38. } 
	\subsection{Задача 1} 
	\textit{Для квадратичной формы 
		\[
		Q(x_1, x_2, x_3) = x_1^2(-3b + 17) + x_2^2(7 - b) + 4x_3^2 + 2x_1x_2(10 - 2b) + 2x_1x_3(3b - 13) + 2x_2x_3(-7 + b)
		\]
	выясните при каких значениях параметра b она является положительно определённой, а при каких -- отрицательно определённой.}

	\textbf{Решение.}\\
	Составим матрицу квадратичной формы. Она будет иметь вид: 
	\[ A=
	\begin{pmatrix}
	(-3b + 17) & (10 - 2b) & (3b - 13) \\
	(10 - 2b) & (7 - b) & (-7 + b) \\
	(3b - 13) & (-7 + b) & 4
	\end{pmatrix}
	\]
	
	По критерию Сильвестра для того, чтобы квадратичная форма была положительно опеределённой нужно чтобы все угловые миноры ее матрицы были положительны. Найдем данные миноры:
	\[
		\Delta_1 = a_{11} = (-3b + 17) > 0 \Rightarrow b < \frac{17}{3}
	\]
	\[
		\Delta_2 = \begin{vmatrix}
		(-3b + 17) & (10 - 2b) \\
		(10 - 2b) & (7 - b)
		\end{vmatrix} > 0
	\]
	\begin{align*}
	&(-3b + 17)(7 - b) - (10 - 2b)^2 = (-21b + 119 + 3b^2 -17b) - \\ 
	& - (100 -40b + 4b^2) = -b^2 + 2b + 19. \\
	\\
	&-b^2 + 2b + 19 = 0 \\
	&D = 4 + 76 = 4\sqrt{5} \\
	&b = \frac{-2 \pm 4\sqrt{5}}{-2} \\
	&b_1 = 1 + 2\sqrt{5},\ b_2 = 1 - 2\sqrt{5}.
	\end{align*}
	Тогда, так как парабола смотрит вниз, $\Delta_2 > 0$ при $1 - 2\sqrt{5} < b < 1 + 2\sqrt{5}$.

	
	\[ \Delta_3 =
		\begin{vmatrix}
		(-3b + 17) & (10 - 2b) & (3b - 13) \\
		(10 - 2b) & (7 - b) & (-7 + b) \\
		(3b - 13) & (-7 + b) & 4
		\end{vmatrix} = 
		\begin{vmatrix}
		(-3b + 17) & (10 - 2b) & (3b - 13) \\
		(-3 + b) & 0 & (-3 + b) \\
		(3b - 13) & (-7 + b) & 4
		\end{vmatrix} = 
	\]
	\[
		= \begin{vmatrix}
		(-6b + 30) & (10 - 2b) & (3b - 13) \\
		0 & 0 & (-3 + b) \\
		(3b - 17) & (-7 + b) & 4
		\end{vmatrix} = 
		\begin{vmatrix}
		-4 & -4 & (3b - 5) \\
		0 & 0 & (-3 + b) \\
		(3b - 17) & (-7 + b) & 4
		\end{vmatrix} = 
	\]
	\[
		= \begin{vmatrix}
		-4 & -4 & 4 \\
		0 & 0 & (-3 + b) \\
		(3b - 17) & (-7 + b) & 4
		\end{vmatrix} = 
		\begin{vmatrix}
		8 & -4 & 4 \\
		0 & 0 & (-3 + b) \\
		4 & (-7 + b) & 4
		\end{vmatrix} = 
	\]
	\[ = (3 - b) \cdot
		\begin{vmatrix}
		8 & -4 \\
		4 & (-7 + b)
		\end{vmatrix} = (3 - b) \cdot (8b - 40) = -8b^2 + 64b - 120 > 0
	\]
	В данном определители порядок преобразований следующий, при этом ни одно из данных элементарных преобразований не изменило определитель:
	\begin{itemize}
		\item Прибавили ко второй строке третью.
		\item Отняли от первого столбца третий.
		\item Прибавили к первой строке третью, домноженную на 2.
		\item Отняли от первой строки вторую, домноженную на 3.
		\item Отнимем от первого столбца второй, домноженый на 3.
	\end{itemize}
	
	\begin{align*}
	&-8b^2 + 64b - 120 = 0 \\
	&D = 4096 - 3840 = 256 = 16^2 \\ 
	&b = \frac{-64 \pm 16}{-16} \\
	&b_1 = 3,\ b_2 = 5.
	\end{align*}
	Тогда, так как парабола смотрит вниз, $\Delta_3 > 0$ при $3 < b < 5$.
	
	Заметим, что $\frac{17}{3} > 1 + 2\sqrt{5}$. 
	Заметим, что интервалы для $\Delta_2$ и $\Delta_3$ входят в интервал для $\Delta_1$. При этом интервал для $\Delta_3$ входит в интервал для $\Delta_2$.
	
	Тогда форма является положительно определённой при $3 < b < 5$.

	По критерию Сильвестра для того, чтобы квадратичная форма была отрицательно опеределённой, нужно чтобы угловые миноры ётного порядка её матрицы были положительны, а нечётного порядка — отрицательны. Тогда:
	\[
		\Delta_1 = a_{11} = (-3b + 17) < 0 \Rightarrow b > \frac{17}{3}
	\]	
	\[
		\Delta_2 > 0\ при\ 1 - 2\sqrt{5} < b < 1 + 2\sqrt{5}.
	\]	
	\[
		\Delta_3 < 0\ при\ b < 3, b > 5
	\]
	Интервал для $\Delta_2$ не входит в интервал для $\Delta_1$.
	Значит нет b, при которых форма является отрицательно определённой.
	
	\subsection{Задача 2} 
	\textit{Подпространство $U$ евклидова пространства в $\mathbb{R}^4$ задано уравнением $3x_1 + 5x_2 - x_3 + 3x_4 = 0.$\\
	(a) Постройте в $U$ ортонормированный базис.\\
	(б) Для вектора $u = (0,2,1,0)$ найдите его проекцию на $U$, его ортогональную составляющую относительно $U$ и расстояние от него до $U$.}

	\textbf{Решение.}\\
	(a) Запишем уравнение в виде $x_3 = 3x_1 + 5x_2 + 3x_4$ и найдем ФСР. Количество решений в ней будет равно 3. \\
	\begin{tabular}[t]{|l|l|l|l|}
		\hline
		$x_1$	& $x_2$ & $x_3$ & $x_4$\\
		\hline
		$1$	& $0$ & $3$ & $0$\\
		\hline
		$0$	& $1$ & $5$ & $0$\\
		\hline
		$0$	& $0$ & $3$ & $1$\\
		\hline
	\end{tabular}
	
	\[\Phi = 
		\begin{pmatrix}
		\phi_1 & \phi_2 & \phi_3 
		\end{pmatrix} = 
		\begin{pmatrix}
		1 & 0 &  0 \\
		0 & 1 &  0 \\
		3 & 5 &  3 \\
		0 & 0 &  1 \\
		\end{pmatrix}
	\]
	Где столбцы $\phi_1, \phi_2, \phi_3$ образуют ФСР и заданное подпространоство будет являться линейной оболочкой данных столбцов, поскольку они линейно-независимы(легко заметить, так как в трех строках в одном месте единица, а в остальных нули). 
	
	Для наглядности запишем векторы, которые образуют подпространство:
	\begin{align*}
		&\vec x_1 =  \begin{pmatrix} 1 & 0 & 3 & 0 \end{pmatrix} \\
		&\vec x_2 =  \begin{pmatrix} 0 & 1 & 5 & 0 \end{pmatrix} \\
		&\vec x_3 =  \begin{pmatrix} 0 & 0 & 3 & 1 \end{pmatrix} \\
	\end{align*}
	
	Теперь с помощью метода Грама-Шмидта ортогонализируем и нормируем данные векторы.\\
	$(\vec{u}, \vec{v})$ -- скалярное произведение векторов.	
	\begin{align*}
	&\vec b_1 = \begin{pmatrix} 1 & 0 & 3 & 0 \end{pmatrix} \\
	&\vec b_2 = \vec x_2 - \frac{(\vec x_2,\vec b_1)}{(\vec b_1,\vec b_1)} \cdot \vec b_1  = \begin{pmatrix} 0 & 1 & 5 & 0 \end{pmatrix} - \frac{15}{10} \cdot \begin{pmatrix} 1 & 0 & 3 & 0 \end{pmatrix} = \begin{pmatrix} -1,5 & 1 & 0,5 & 0 \end{pmatrix} \\
	&\vec b_3 = \vec x_3 - \frac{(\vec x_3,\vec b_1)}{(\vec b_1,\vec b_1)} \cdot \vec b_1 - \frac{(\vec x_3,\vec b_2)}{(\vec b_2,\vec b_2)} \cdot \vec b_2  = \begin{pmatrix} 0 & 0 & 3 & 1 \end{pmatrix} - \frac{9}{10} \begin{pmatrix} 1 & 0 & 3 & 0 \end{pmatrix} - \\
	& - \frac{1,5}{3,5} \begin{pmatrix} -1,5 & 1 & 0,5 & 0 \end{pmatrix} = \begin{pmatrix} -\frac{3,15}{3,5} & 0 & \frac{1,05}{3,5} & 1 \end{pmatrix} - \begin{pmatrix} -\frac{2,25}{3,5} & \frac{1,5}{3,5} & \frac{0,75}{3,5} & 0 \end{pmatrix} = \\
	& = \frac{1}{3,5}\begin{pmatrix} -0,9 & -1,5 & 0,3 & 3,5 \end{pmatrix}
	\end{align*}
	
	Нормируем их по формуле
	\[
		\vec e_j = \frac{\vec b_j}{||\vec b_j||},\ где ||\vec b_j|| = \sqrt{(b_j, b_j)}.
	\]
	
	\begin{align*}
	&e_1 = \frac{\vec b_1}{\sqrt{10}} = \begin{pmatrix} \frac{1}{\sqrt{10}} & 0 & \frac{3}{\sqrt{10}} & 0 \end{pmatrix} \\
	&e_2 = \frac{\vec b_2}{\sqrt{3,5}} = \begin{pmatrix} -\frac{1,5}{\sqrt{3,5}} & \frac{1}{\sqrt{3,5}} & \frac{0,5}{\sqrt{3,5}} & 0 \end{pmatrix} = \begin{pmatrix} -\frac{3}{\sqrt{14}} & \sqrt{\frac{2}{7}} & \frac{1}{\sqrt{14}} & 0 \end{pmatrix}\\
	&e_3 = \frac{\vec b_3}{\sqrt{\frac{44}{35}}}= \frac{1}{\sqrt{\frac{44}{35}} \cdot 3,5}\begin{pmatrix} -0,9 & -1,5 & 0,3 & 3,5 \end{pmatrix} = \frac{1}{\sqrt{15,4}}\begin{pmatrix} -0,9 & -1,5 & 0,3 & 3,5 \end{pmatrix}
	\end{align*}
	$(e_1, e_2, e_3)$ -- искомый ортонормированный базис.
	
	(б) Ортонормированный базис $U$ 
	\begin{align*}
	&e_1 = \begin{pmatrix} \frac{1}{\sqrt{10}} & 0 & \frac{3}{\sqrt{10}} & 0 \end{pmatrix} \\
	&e_2 = \begin{pmatrix} -\frac{3}{\sqrt{14}} & \sqrt{\frac{2}{7}} & \frac{1}{\sqrt{14}} & 0 \end{pmatrix}\\
	&e_3 = \frac{1}{\sqrt{15,4}}\begin{pmatrix} -0,9 & -1,5 & 0,3 & 3,5 \end{pmatrix}
	\end{align*}
	вектор $u = (0,2,1,0)$.
	
	Учитывая, что у мы нашли ортонормированный базис $U$ ортогональная проекция будет иметь вид:
	\[
	pr_Uu = \sum_{i=1}^{3} (u, e_i)e_i
	\]
	\begin{align*}
	&pr_Uu = \frac{3}{\sqrt{10}}\begin{pmatrix} \frac{1}{\sqrt{10}} & 0 & \frac{3}{\sqrt{10}} & 0 \end{pmatrix} + \frac{5}{\sqrt{14}} \begin{pmatrix} -\frac{3}{\sqrt{14}} & \sqrt{\frac{2}{7}} & \frac{1}{\sqrt{14}} & 0 \end{pmatrix} -\\
	&- \frac{2,7}{15,4} \begin{pmatrix} -0,9 & -1,5 & 0,3 & 3,5 \end{pmatrix} = \begin{pmatrix} 0,3 & 0 & 0,9 & 0 \end{pmatrix} + \begin{pmatrix} -\frac{15}{14} & \frac{5}{7} & \frac{5}{14} & 0 \end{pmatrix} +\\
	&+ \begin{pmatrix} \frac{2,43}{15,4} & \frac{4,05}{15,4} & -\frac{0,81}{15,4} & -\frac{9.45}{15,4} \end{pmatrix} = \begin{pmatrix} -\frac{27}{44} & \frac{43}{44} & \frac{53}{44} & -\frac{27}{44} \end{pmatrix} = \frac{1}{44} \begin{pmatrix} -27 & 43 & 53 & -27 \end{pmatrix}
	\end{align*}
	Тогда найдем ортогональную составляющую:
	\[
	ort_Uu = \begin{pmatrix} \frac{27}{44} & \frac{45}{44} & -\frac{9}{44} & \frac{27}{44} \end{pmatrix} = \frac{1}{44} \begin{pmatrix} 27 & 45 & -9 & 27 \end{pmatrix}
	\]
	Заметим, что $(pr_Uu, ort_Uu) = 0$, значит они найдены правильно.
	
	Расстояние от $\vec u$ до подпространства $U$ равно $|ort_Uu| = \frac{9}{\sqrt{44}}$.
	
	
	\subsection{Задача 3} 
	\textit{Составьте уравнение прямой в $\mathbb{R}^3$, параллельной плоскости $-2x+4y+2z=0$, проходящей через точку $(2,3,2)$ и пересекающей прямую $x = 2t + 1,\ y = 4t - 2, z = 3t + 1$.}
	
	\textbf{Решение.}\\
	Заметим, что искомая прямая будет лежать в некой плоскости, которая должна быть параллельна плоскости, заданной уравнением $-2x+4y+2z=0$, и при этом проходящей через точку $(2,3,2)$.
	
	Тогда данная плоскость будет задана уравнением $-2(x - 2) + 3(y - 3) + 2(z - 2) = 0$. 
	
	Теперь рассмотрим прямую:
	\[
		\begin{cases}
			x = 2t + 1 \\
			y = 4t - 2 \\
			z = 3t + 1
		\end{cases}
	\]
	или:
	\[
		\begin{cases}
			2x - y = 4 \\
			3y - 4z = -10 
		\end{cases}
	\]
	
	Тогда найдем точку пересечения заданной прямой и плоскости и нам останется просто записать уравнение прямой, проходящей через две точки. Из уравнения плоскости: $-2x + 4 + 3y - 9 + 2z - 4 = 0 \Rightarrow -2x + 3y + 2z = 9$.
	\[
		\begin{cases}
			2x - y = 4 \\
			3y - 4z = -10 \\
			-2x + 3y + 2z = 9
		\end{cases}
	\]
	
	Решим данную систему с помощью метода Гаусса: 
	\[
		\begin{pmatrix} 
			2 & -1 & 0 & | & 4 \\
			0 & 3 & -4 & | & -10 \\
			-2 & 3 & 2 & | & 9 
		\end{pmatrix} \rightarrow
		\begin{pmatrix} 
		2 & -1 & 0 & | & 4 \\
		0 & 3 & -4 & | & -10 \\
		0 & 6 & 6 & | & 39 
		\end{pmatrix} \rightarrow
		\begin{pmatrix} 
		2 & -1 & 0 & | & 4 \\
		0 & 3 & -4 & | & -10 \\
		0 & 0 & 14 & | & 59 
		\end{pmatrix} 
	\]
	\begin{align*}
		&z = \frac{59}{14}. \\
		&y = \frac{\frac{59 \cdot 4}{14} - 10}{3} = \frac{16}{7}. \\
		&x = \frac{\frac{16}{7} + 4}{2} = \frac{22}{7}.
	\end{align*} 
	Теперь у нас есть две точки, через которые проходит прямая, уравнение которой мы ищем: $A=(2,3,2)$ и $B=(\frac{22}{7}, \frac{16}{7}, \frac{59}{14})$.
	
	Направляющим вектором данной прямой будет являться $\overrightarrow{AB} = (\frac{8}{7}, -\frac{5}{7}, \frac{31}{14})$, тогда параметрическое уравнение прямой будет иметь вид: 
	\[
		\begin{cases}
			x = \frac{8}{7}t + 2 \\
			y = -\frac{5}{7}t + 3 \\
			z = \frac{31}{14}t + 2
		\end{cases}
	\]
	
	
	\subsection{Задача 4} 
	\textit{Дан куб $ABCDA'B'C'D'$ со стороной 3. Точка $F$ -- середина ребра $BB'$, а точка $E$ лежит на ребре $BB'$, причём $BE : EB' = 5 : 6$. Найдите угол и расстояние между прямыми $AE$ и $D'F$.}
	
	\textbf{Решение.}\\
	Расположим наш куб в декартовой системе координат следующим образом:
	\begin{align*}
		&A = (0,0,0) \\
		&B = (0,3,0) \\
		&C = (3,3,0) \\
		&D = (3,0,0) \\
		&A' = (0,0,3) \\
		&B' = (0,3,3) \\
		&C' = (3,3,3) \\
		&D' = (3,0,3) \\
	\end{align*}
	Тогда точка $F$ будет иметь координаты $(0, 3, 1.5)$, а точка $E$ -- $(0, 3, \frac{18}{11})$.
	
	Найдем уравнение прямых $AE$ и $DF$:\\
	Точка на прямой $AE$ -- $A(0,0,0)$ и направляющий вектор $11\overrightarrow{AE} = (0,33,18)$, значит ее уравнение: 
	\[
		\frac{x}{0} = \frac{y}{33} = \frac{z}{18}
	\]
	Точка на прямой $D'F$ -- $F(0, 3, 1.5)$ и направляющий вектор $\overrightarrow{D'F} = (-3,3,-1.5)$, значит ее уравнение: 
	\[
		\frac{x}{-3} = \frac{y - 3}{3} = \frac{z - 1,5}{-1,5}
	\]
	
	Найдем угол между данными прямыми по формуле: 
	
	\[
		cos\ \phi = \frac{(11\overrightarrow{AE}, \overrightarrow{D'F})}{\sqrt{|11\overrightarrow{AE}}| \cdot \sqrt{\overrightarrow{D'F}}|} = \frac{72}{\sqrt{1413}\cdot \sqrt{20,25}} = \frac{16}{\sqrt{1413}}.
	\]
	\[
		\phi = arccos\left(\frac{16}{\sqrt{1413}}\right).
	\]
	
	Теперь найдем расстояние между данными скрещивающимися прямыми. 

	\end{document}