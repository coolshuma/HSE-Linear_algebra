\documentclass[a4paper,12pt]{article}
\usepackage{amsmath}
\usepackage{cmap}					% поиск в PDF
\usepackage{mathtext} 				% русские буквы в формулах
\usepackage[T2A]{fontenc}			% кодировка
\usepackage[utf8]{inputenc}			% кодировка исходного текста
\usepackage[english,russian]{babel}	% локализация и переносы
\usepackage{amssymb}				    % Для красивого (!) \mathbb с  буквами и цифрами
\usepackage{mathbbol}

% Изменим формат \section и \subsection:
\usepackage{titlesec}
\titleformat{\section}
{\vspace{1cm}\centering\LARGE\bfseries}	% Стиль заголовка
{}										% префикс
{0pt}									% Расстояние между префиксом и заголовком
{} 										% Как отображается префикс
\titleformat{\subsection}				% Аналогично для \subsection
{\Large\bfseries}
{}
{0pt}
{}

%% Отступы между абзацами и в начале абзаца 
\setlength{\parindent}{0pt}
\setlength{\parskip}{\medskipamount}
\begin{document}
	\section{Индивидульное домашнее задание 5 \\ Шумилкин Андрей. Группа 165 \\ Вариант 38. } 
	\subsection{Задача 1} 
	\textit{Для квадратичной формы 
		\[
		Q(x_1, x_2, x_3) = x_1^2(-3b + 17) + x_2^2(7 - b) + 4x_3^2 + 2x_1x_2(10 - 2b) + 2x_1x_3(3b - 13) + 2x_2x_3(-7 + b)
		\]
	выясните при каких значениях параметра b она является положительно определённой, а при каких -- отрицательно определённой.}

	\textbf{Решение.}\\
	Составим матрицу квадратичной формы. Она будет иметь вид: 
	\[ A=
	\begin{pmatrix}
	(-3b + 17) & (10 - 2b) & (3b - 13) \\
	(10 - 2b) & (7 - b) & (-7 + b) \\
	(3b - 13) & (-7 + b) & 4
	\end{pmatrix}
	\]
	
	По критерию Сильвестра для того, чтобы квадратичная форма была положительно опеределённой нужно чтобы все угловые миноры ее матрицы были положительны. Найдем данные миноры:
	\[
		\Delta_1 = a_{11} = (-3b + 17) > 0 \Rightarrow b < \frac{17}{3}
	\]
	\[
		\Delta_2 = \begin{vmatrix}
		(-3b + 17) & (10 - 2b) \\
		(10 - 2b) & (7 - b)
		\end{vmatrix} > 0
	\]
	\begin{align*}
	&(-3b + 17)(7 - b) - (10 - 2b)^2 = (-21b + 119 + 3b^2 -17b) - \\ 
	& - (100 -40b + 4b^2) = -b^2 + 2b + 19. \\
	\\
	&-b^2 + 2b + 19 = 0 \\
	&D = 4 + 76 = 4\sqrt{5} \\
	&b = \frac{-2 \pm 4\sqrt{5}}{-2} \\
	&b_1 = 1 + 2\sqrt{5},\ b_2 = 1 - 2\sqrt{5}.
	\end{align*}
	Тогда, так как парабола смотрит вниз, $\Delta_2 > 0$ при $1 - 2\sqrt{5} < b < 1 + 2\sqrt{5}$.

	
	\[ \Delta_3 =
		\begin{vmatrix}
		(-3b + 17) & (10 - 2b) & (3b - 13) \\
		(10 - 2b) & (7 - b) & (-7 + b) \\
		(3b - 13) & (-7 + b) & 4
		\end{vmatrix} = 
		\begin{vmatrix}
		(-3b + 17) & (10 - 2b) & (3b - 13) \\
		(-3 + b) & 0 & (-3 + b) \\
		(3b - 13) & (-7 + b) & 4
		\end{vmatrix} = 
	\]
	\[
		= \begin{vmatrix}
		(-6b + 30) & (10 - 2b) & (3b - 13) \\
		0 & 0 & (-3 + b) \\
		(3b - 17) & (-7 + b) & 4
		\end{vmatrix} = 
		\begin{vmatrix}
		-4 & -4 & (3b - 5) \\
		0 & 0 & (-3 + b) \\
		(3b - 17) & (-7 + b) & 4
		\end{vmatrix} = 
	\]
	\[
		= \begin{vmatrix}
		-4 & -4 & 4 \\
		0 & 0 & (-3 + b) \\
		(3b - 17) & (-7 + b) & 4
		\end{vmatrix} = 
		\begin{vmatrix}
		8 & -4 & 4 \\
		0 & 0 & (-3 + b) \\
		4 & (-7 + b) & 4
		\end{vmatrix} = 
	\]
	\[ = (3 - b) \cdot
		\begin{vmatrix}
		8 & -4 \\
		4 & (-7 + b)
		\end{vmatrix} = (3 - b) \cdot (8b - 40) = -8b^2 + 64b - 120 > 0
	\]
	В данном определители порядок преобразований следующий, при этом ни одно из данных элементарных преобразований не изменило определитель:
	\begin{itemize}
		\item Прибавили ко второй строке третью.
		\item Отняли от первого столбца третий.
		\item Прибавили к первой строке третью, домноженную на 2.
		\item Отняли от первой строки вторую, домноженную на 3.
		\item Отнимем от первого столбца второй, домноженый на 3.
	\end{itemize}
	
	\begin{align*}
	&-8b^2 + 64b - 120 = 0 \\
	&D = 4096 - 3840 = 256 = 16^2 \\ 
	&b = \frac{-64 \pm 16}{-16} \\
	&b_1 = 3,\ b_2 = 5.
	\end{align*}
	Тогда, так как парабола смотрит вниз, $\Delta_3 > 0$ при $3 < b < 5$.
	
	Заметим, что $\frac{17}{3} > 1 + 2\sqrt{5}$. 
	Заметим, что интервалы для $\Delta_2$ и $\Delta_3$ входят в интервал для $\Delta_1$. При этом интервал для $\Delta_3$ входит в интервал для $\Delta_2$.
	
	Тогда форма является положительно определённой при $3 < b < 5$.

	По критерию Сильвестра для того, чтобы квадратичная форма была отрицательно опеределённой, нужно чтобы угловые миноры ётного порядка её матрицы были положительны, а нечётного порядка — отрицательны. Тогда:
	\[
		\Delta_1 = a_{11} = (-3b + 17) < 0 \Rightarrow b > \frac{17}{3}
	\]	
	\[
		\Delta_2 > 0\ при\ 1 - 2\sqrt{5} < b < 1 + 2\sqrt{5}.
	\]	
	\[
		\Delta_3 < 0\ при\ b < 3, b > 5
	\]
	Интервал для $\Delta_2$ не входит в интервал для $\Delta_1$.
	Значит нет b, при которых форма является отрицательно определённой.
	
	\subsection{Задача 2} 
	\textit{Подпространство $U$ евклидова пространства в $\mathbb{R}^4$ задано уравнением $3x_1 + 5x_2 - x_3 + 3x_4 = 0.$\\
	(a) Постройте в $U$ ортонормированный базис.\\
	(б) Для вектора $u = (0,2,1,0)$ найдите его проекцию на $U$, его ортогональную составляющую относительно $U$ и расстояние от него до $U$.}

	\textbf{Решение.}\\
	Запишем уравнение в виде $x_3 = 3x_1 + 5x_2 + 3x_4$ и найдем ФСР. Количество решений в ней будет равно 3. \\
	\begin{tabular}[t]{|l|l|l|l|}
		\hline
		$x_1$	& $x_2$ & $x_3$ & $x_4$\\
		\hline
		$1$	& $0$ & $3$ & $0$\\
		\hline
		$0$	& $1$ & $5$ & $0$\\
		\hline
		$0$	& $0$ & $3$ & $1$\\
		\hline
	\end{tabular}
	
	\subsection{Задача 3} 
	\textit{Составьте уравнение прямой в $\mathbb{R}^3$, параллельной плоскости $-2x+4y+2z=0$, проходящей через точку $(2,3,2)$ и пересекающей прямую $x = 2t + 1,\ y = 4t - 2, z = 3t + 1$.}
	
	\textbf{Решение.}\\
	
	\subsection{Задача 4} 
	\textit{Дан куб $ABCDA'B'C'D'$ со стороной 3. Точка $F$ -- середина ребра $BB'$, а точка $E$ лежит на ребре $BB'$, причём $BE : EB' = 5 : 6$. Найдите угол и расстояние между прямыми $AE$ и $D'F$.}
	
	\textbf{Решение.}\\

	\end{document}