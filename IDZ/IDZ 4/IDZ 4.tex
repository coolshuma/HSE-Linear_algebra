\documentclass[a4paper,12pt]{article}
\usepackage{amsmath}
\usepackage{cmap}					% поиск в PDF
\usepackage{mathtext} 				% русские буквы в формулах
\usepackage[T2A]{fontenc}			% кодировка
\usepackage[utf8]{inputenc}			% кодировка исходного текста
\usepackage[english,russian]{babel}	% локализация и переносы
\usepackage{amssymb}				    % Для красивого (!) \mathbb с  буквами и цифрами
\usepackage{mathbbol}

% Изменим формат \section и \subsection:
\usepackage{titlesec}
\titleformat{\section}
{\vspace{1cm}\centering\LARGE\bfseries}	% Стиль заголовка
{}										% префикс
{0pt}									% Расстояние между префиксом и заголовком
{} 										% Как отображается префикс
\titleformat{\subsection}				% Аналогично для \subsection
{\Large\bfseries}
{}
{0pt}
{}

%% Отступы между абзацами и в начале абзаца 
\setlength{\parindent}{0pt}
\setlength{\parskip}{\medskipamount}
\begin{document}
	\section{Индивидульное домашнее задание 4 \\ Шумилкин Андрей. Группа 163 \\ Вариант 44. } 
	\subsection{Задача 1} 
	Базис $\mathbb{e} = (e_1, e_2, e_3)$ и базис $\mathbb{e}^\prime = (e_1^\prime, e_2^\prime, e_3^\prime)$.
	
	$e_1 = (1, 2, 0)$, $e_2 = (2, -1, 1)$, $e_3 = (0, 1, 2)$. \\
	$e_1^\prime = (-2, 3, 3)$, $e_2^\prime = (4, -1, 4)$, $e_3^\prime = (1, -4, -1)$. \\
	
	$\vec v$ имеет в базисе $\mathbb{e}$ координаты $(1, 4, 2)$.
	
	а)
	Матрица перехода $C$ -- это такая матрица, что $(e_1^\prime, e_2^\prime, e_3^\prime) = (e_1, e_2, e_3) \cdot C$ по определению, причем здесь запись в скобках значит, это матрица в которой указанные базисные векторы будут столбцами в указанном порядке\\
	Тогда $C = (e_1, e_2, e_3)^{-1} \cdot (e_1^\prime, e_2^\prime, e_3^\prime)$. \\
	Решим данное матричное уравнение: 
	\[
	\begin{pmatrix}
	1 & 2 & 0 & | & -2 & 4 & 1 \\
	2 & -1 & 1 & | & 3 & -1 & -4 \\
	0 & 1 & 2 & | & 3 & 4 & -1 \\
	\end{pmatrix}
	\]
	
	Добавим к 2-ой строке 1-ую, домноженную на -2:
	
	\[
	\begin{pmatrix}
	1 & 2 & 0 & | & -2 & 4 & 1 \\
	0 & -5 & 1 & | & 7 & -9 & -6 \\
	0 & 1 & 2 & | & 3 & 4 & -1 \\
	\end{pmatrix}
	\]
	
	Добавим к 2-ой строке 3-ую, домноженную на 4:
	
	\[
	\begin{pmatrix}
	1 & 2 & 0 & | & -2 & 4 & 1 \\
	0 & -1 & 9 & | & 19 & 7 & -10 \\
	0 & 1 & 2 & | & 3 & 4 & -1 \\
	\end{pmatrix}
	\]
	
	Добавим к 3-ой строке 2-ую, домноженную на 1:
	
	\[
	\begin{pmatrix}
	1 & 2 & 0 & | & -2 & 4 & 1 \\
	0 & -1 & 9 & | & 19 & 7 & -10 \\
	0 & 0 & 11 & | & 22 & 11 & -11 \\
	\end{pmatrix}
	\]
	
	Домножим 2-ую строку на -1:
	
	\[
	\begin{pmatrix}
	1 & 2 & 0 & | & -2 & 4 & 1 \\
	0 & 1 & -9 & | & -19 & -7 & 10 \\
	0 & 0 & 11 & | & 22 & 11 & -11 \\
	\end{pmatrix}
	\]
	
	Разделим 3-ую строку на 11:
	
	\[
	\begin{pmatrix}
	1 & 2 & 0 & | & -2 & 4 & 1 \\
	0 & 1 & -9 & | & -19 & -7 & 10 \\
	0 & 0 & 1 & | & 2 & 1 & -1 \\
	\end{pmatrix}
	\]
	
	Добавим к 2-ой строке 3-ую, домноженную на 9:
	
	\[
	\begin{pmatrix}
	1 & 2 & 0 & | & -2 & 4 & 1 \\
	0 & 1 & 0 & | & -1 & 2 & 1 \\
	0 & 0 & 1 & | & 2 & 1 & -1 \\
	\end{pmatrix}
	\]
	
	Добавим к 1-ой строке 2-ую, домноженную на -2:
	
	\[
	\begin{pmatrix}
	1 & 0 & 0 & | & 0 & 0 & -1 \\
	0 & 1 & 0 & | & -1 & 2 & 1 \\
	0 & 0 & 1 & | & 2 & 1 & -1 \\
	\end{pmatrix}
	\]
	
	Тогда матрица перехода от базиса $\mathbb{e}$ к базису $\mathbb{e}^\prime$: 
	\[ C =
	\begin{pmatrix}
	0 & 0 & -1 \\
	-1 & 2 & 1 \\
	2 & 1 & -1 \\
	\end{pmatrix}
	\]
	
	б) 
	Как мы знаем, матрица перехода, умноженная(матрица перехода слева) на столбец, составленный из координат вектора в базисе $\mathbb{e}^\prime$, даст столбец координат данного вектора в базисе $\mathbb{e}$. \\
	Тогда найти координаты $\vec v$ в базисе $\mathbb{e}^\prime$ мы можем следующим образом(в данном месте оговорим, что $v$ -- вектор, потому что стрелка с чертой после нее выглядит не очень красиво, сливается с ней): 
	\[
		 (v^{\prime})^T = C^{-1} \cdot (v)^T
	\]
	
	Найдем $C^{-1}$:
	\[
	\begin{pmatrix}
	0 & 0 & -1 & | & 1 & 0 & 0 \\
	-1 & 2 & 1 & | & 0 & 1 & 0 \\
	2 & 1 & -1 & | & 0 & 0 & 1 \\
	\end{pmatrix}
	\]
	
	Добавим к 1-ой строке 2-ую, домноженную на -1:
	
	\[
	\begin{pmatrix}
	1 & -2 & -2 & | & 1 & -1 & 0 \\
	-1 & 2 & 1 & | & 0 & 1 & 0 \\
	2 & 1 & -1 & | & 0 & 0 & 1 \\
	\end{pmatrix}
	\]
	
	Добавим к 2-ой строке 1-ую, домноженную на 1:
	
	\[
	\begin{pmatrix}
	1 & -2 & -2 & | & 1 & -1 & 0 \\
	0 & 0 & -1 & | & 1 & 0 & 0 \\
	2 & 1 & -1 & | & 0 & 0 & 1 \\
	\end{pmatrix}
	\]
	
	Добавим к 3-ой строке 1-ую, домноженную на -2:
	
	\[
	\begin{pmatrix}
	1 & -2 & -2 & | & 1 & -1 & 0 \\
	0 & 0 & -1 & | & 1 & 0 & 0 \\
	0 & 5 & 3 & | & -2 & 2 & 1 \\
	\end{pmatrix}
	\]
	
	Добавим к 2-ой строке 3-ую, домноженную на 1:
	
	\[
	\begin{pmatrix}
	1 & -2 & -2 & | & 1 & -1 & 0 \\
	0 & 5 & 2 & | & -1 & 2 & 1 \\
	0 & 5 & 3 & | & -2 & 2 & 1 \\
	\end{pmatrix}
	\]
	
	Добавим к 3-ой строке 2-ую, домноженную на -1:
	
	\[
	\begin{pmatrix}
	1 & -2 & -2 & | & 1 & -1 & 0 \\
	0 & 5 & 2 & | & -1 & 2 & 1 \\
	0 & 0 & 1 & | & -1 & 0 & 0 \\
	\end{pmatrix}
	\]
	
	Добавим к 2-ой строке 3-ую, домноженную на -2:
	
	\[
	\begin{pmatrix}
	1 & -2 & -2 & | & 1 & -1 & 0 \\
	0 & 5 & 0 & | & 1 & 2 & 1 \\
	0 & 0 & 1 & | & -1 & 0 & 0 \\
	\end{pmatrix}
	\]
	
	Добавим к 1-ой строке 3-ую, домноженную на 2:
	
	\[
	\begin{pmatrix}
	1 & -2 & 0 & | & -1 & -1 & 0 \\
	0 & 5 & 0 & | & 1 & 2 & 1 \\
	0 & 0 & 1 & | & -1 & 0 & 0 \\
	\end{pmatrix}
	\]
	
	Разделим 2-ую строку на 5:
	
	\[
	\begin{pmatrix}
	1 & -2 & 0 & | & -1 & -1 & 0 \\
	0 & 1 & 0 & | & 0.2 & 0.4 & 0.2 \\
	0 & 0 & 1 & | & -1 & 0 & 0 \\
	\end{pmatrix}
	\]
	
	Добавим к 1-ой строке 2-ую, домноженную на 2:
	
	\[
	\begin{pmatrix}
	1 & 0 & 0 & | & -0.6 & -0.2 & 0.4 \\
	0 & 1 & 0 & | & 0.2 & 0.4 & 0.2 \\
	0 & 0 & 1 & | & -1 & 0 & 0 \\
	\end{pmatrix}
	\]
	
	\[ C^{-1} = 
	\begin{pmatrix}
	-0.6 & -0.2 & 0.4 \\
	0.2 & 0.4 & 0.2 \\
	-1 & 0 & 0 \\
	\end{pmatrix}
	\]
	
	После умножим в указанном нами порядке:
	\[
	\begin{pmatrix}
	-0.6 & -0.2 & 0.4 \\
	0.2 & 0.4 & 0.2 \\
	-1 & 0 & 0 \\
	\end{pmatrix} \cdot 
	\begin{pmatrix}
	1 \\
	4 \\
	2 \\
	\end{pmatrix} = 
	\begin{pmatrix}
	-0.6 \\
	2.2 \\
	-1 \\
	\end{pmatrix}
	\]

	Получаем, что координаты данного вектора в базисе $\mathbb{e}^\prime$ равны $(-0.6, 2.2, -1)$.
	
	\subsection{Задача 2} 
	$a_1 = (1, 0, 0, 0, 0)$, $a_2 = (0, 1, 0, 0, 0)$, $a_3 = (0, 0, 5, 0, 0)$, $a_4 = (-3, 0, 0, 1, 0)$, $a_5 = (-2, 0, 0, 0, 1)$. \\
	
	$b_1 = (1, 0, 0)$, $b_2 = (0, 1, -2)$, $b_3 = (10, -10, 20)$, $b_4 = (-6, 0, 0)$, $b_5 = (-1, -1, 2)$.
	
	а)
	Мы знаем, что если $V, W$ -- векторные пространства над полем $F$ и $e_1, e_2, \ldots, e_n$ -- базис $V$, то для всякого набора векторов $F$ и $f_1, f_2, \ldots, f_n$ существует единственное линейное отображение $\varphi$ : $V \rightarrow W$ такое, что $\varphi(e_1) = f_1,\ \varphi(e_2) = f_2, \ldots,\ \varphi(e_n) = f_n$. \\
	Откуда следует, что, если мы докажем, что $(a_1, a_2, a_3, a_4, a_5)$ является базисом в $\mathbb{R}^5$, то мы докажем, что искомое отображение существует и единственно. \\
	Для этого просто проверим, что ранг матрицы, составленной из данных векторов равен пяти: 
	\[
	\begin{pmatrix}
	1 & 0 & 0 & 0 & 0 \\
	0 & 1 & 0 & 0 & 0 \\
	0 & 0 & 5 & 0 & 0 \\
	-3 & 0 & 0 & 1 & 0 \\
	-2 & 0 & 0 & 0 & 1 \\
	\end{pmatrix}
	\]
	
	Добавим к 4-ой строке 1-ую, домноженную на 3:
	
	\[
	\begin{pmatrix}
	1 & 0 & 0 & 0 & 0 \\
	0 & 1 & 0 & 0 & 0 \\
	0 & 0 & 5 & 0 & 0 \\
	0 & 0 & 0 & 1 & 0 \\
	-2 & 0 & 0 & 0 & 1 \\
	\end{pmatrix}
	\]
	
	Добавим к 5-ой строке 1-ую, домноженную на 2:
	
	\[
	\begin{pmatrix}
	1 & 0 & 0 & 0 & 0 \\
	0 & 1 & 0 & 0 & 0 \\
	0 & 0 & 5 & 0 & 0 \\
	0 & 0 & 0 & 1 & 0 \\
	0 & 0 & 0 & 0 & 1 \\
	\end{pmatrix}
	\]
	
	Разделим 3-ую строку на 5:
	
	\[
	\begin{pmatrix}
	1 & 0 & 0 & 0 & 0 \\
	0 & 1 & 0 & 0 & 0 \\
	0 & 0 & 1 & 0 & 0 \\
	0 & 0 & 0 & 1 & 0 \\
	0 & 0 & 0 & 0 & 1 \\
	\end{pmatrix}
	\]
	Да, ранг равен пяти, так что искомое линейное отображение существует и единственно.
	
	б) Из определения ядра линейного отображения, которое является множеством всех векторов из $\mathbb{R}^5$, которые данное линейное отображение переводит в нулевой вектор, и из того, что матрица линейного отображения в базисах $(a_1, a_2, a_3, a_4, a_5)$ и базисе $\mathbb{R}^3$ равна матрице, составленной из векторов $(b_1, b_2, b_3, b_4, b_5)$, записанных по столбцам в указанном порядке, понятно, что ядро будет равно пространству решений ОСЛУ, матрица которого будет равна вышеупомянутой матрице. Тогда решим данную ОСЛУ и найдем ее ФСР, которая и будет базисом ядра: \\
	\[
	\begin{pmatrix}
	1 & 0 & 10 & -6 & -1 \\
	0 & 1 & -10 & 0 & -1 \\
	0 & -2 & 20 & 0 & 2 \\
	\end{pmatrix}
	\]
	
	Добавим к 3-ой строке 2-ую, домноженную на 2:
	
	\[
	\begin{pmatrix}
	1 & 0 & 10 & -6 & -1 \\
	0 & 1 & -10 & 0 & -1 \\
	0 & 0 & 0 & 0 & 0 \\
	\end{pmatrix}
	\]
	
	Получаем:
	\[
	\begin{cases}
	x_1 = -10x_3 + 6x_4 + x_5 \\
	x_2 = 10x_3 + x_5 
	\end{cases}
	\]
	
	Составим ФСР, в которой будет три вектора, а значит и размерность искомого базиса будет равна трем, так как кол-во свободных переменных -- три, взяв значения $(1,0, 0)$, $(1, 0, 1)$ и $(1, 1, 1)$ для $x_3, x_4, x_5$ соответственно. 
	
	Получим три следующих вектора, составляющих базис ядра:
	\[ X_1 = 
	\begin{pmatrix}
	-10 \\
	10  \\
	1 \\
	0 \\
	0 \\
	\end{pmatrix},\ 
	X_2 = 
	\begin{pmatrix}
	6 \\
	0 \\
	0 \\
	1 \\
	0 \\
	\end{pmatrix},\ 
	X_3 = 
	\begin{pmatrix}
	1 \\
	1  \\
	0 \\
	0 \\
	1 \\
	\end{pmatrix}
	\]
	
	Но они записаны в базисе $(a_1, a_2, a_3, a_4, a_5)$, поэтому переведем их в стандартный базис $\mathbb{R}^5$.

	
	\[ u_1 = -10a_1 + 10a_2 + a_3 =  
	\begin{pmatrix}
	-10 \\
	10  \\
	5 \\
	0 \\
	0 \\
	\end{pmatrix}, \]
	
	\[u_2 = 6a_1 + a_4 =  
	\begin{pmatrix}
	3 \\
	0  \\
	0 \\
	1 \\
	0 \\
	\end{pmatrix},\]
	
	\[u_3 = a_1 + a_2 + a_5 =  
	\begin{pmatrix}
	-1 \\
	1 \\
	0 \\
	0 \\
	1  \\
	\end{pmatrix}
	\] \\\\
	
	Теперь найдем базис образа. Образ линейного отображения  в данном случае совпадает с пространством, порожденным векторами-столбцами матрицы линейного отображения(которую мы уже определили выше), поскольку $(a_1, a_2, a_3, a_4, a_5)$ -- базис в $\mathbb{R}^5$.\\
	Тогда нам нужно транспонированную матрицу линейного отображения привести к ступенчатому виду и ее ненулевые строки и будут являться базисом. 
	
	\[
	\begin{pmatrix}
	1 & 0 & 0 \\
	0 & 1 & -2 \\
	10 & -10 & 20 \\
	-6 & 0 & 0 \\
	-1 & -1 & 2 \\
	\end{pmatrix}
	\]
	
	Добавим к 3-ой строке 2-ую, домноженную на 2:
	
	\[
	\begin{pmatrix}
	1 & 0 & 0 \\
	0 & 1 & -2 \\
	10 & -8 & 16 \\
	-6 & 0 & 0 \\
	-1 & -1 & 2 \\
	\end{pmatrix}
	\]
	
	Добавим к 3-ой строке 1-ую, домноженную на -10:
	
	\[
	\begin{pmatrix}
	1 & 0 & 0 \\
	0 & 1 & -2 \\
	0 & -8 & 16 \\
	-6 & 0 & 0 \\
	-1 & -1 & 2 \\
	\end{pmatrix}
	\]
	
	Добавим к 4-ой строке 1-ую, домноженную на 6:
	
	\[
	\begin{pmatrix}
	1 & 0 & 0 \\
	0 & 1 & -2 \\
	0 & -8 & 16 \\
	0 & 0 & 0 \\
	-1 & -1 & 2 \\
	\end{pmatrix}
	\]
	
	Добавим к 5-ой строке 1-ую, домноженную на 1:
	
	\[
	\begin{pmatrix}
	1 & 0 & 0 \\
	0 & 1 & -2 \\
	0 & -8 & 16 \\
	0 & 0 & 0 \\
	0 & -1 & 2 \\
	\end{pmatrix}
	\]
	
	Добавим к 3-ой строке 2-ую, домноженную на 8:
	
	\[
	\begin{pmatrix}
	1 & 0 & 0 \\
	0 & 1 & -2 \\
	0 & 0 & 0 \\
	0 & 0 & 0 \\
	0 & -1 & 2 \\
	\end{pmatrix}
	\]
	
	Добавим к 5-ой строке 2-ую, домноженную на 1:
	
	\[
	\begin{pmatrix}
	1 & 0 & 0 \\
	0 & 1 & -2 \\
	0 & 0 & 0 \\
	0 & 0 & 0 \\
	0 & 0 & 0 \\
	\end{pmatrix}
	\]
	
	Получается, что векторы $(b_1, b_2)$ будут образовывать базис образа линейного отображения и они уже записаны в стандартном базисе $\mathbb{R}^3$. Заметим также, что его размерность будет равна двум, а размерность базиса ядра -- трем, что соответствует теореме о том, что  сумма размерностей двух данных базисов будет равна размерности базиса пространства из которого строится отображение, т.е. $\mathbb{R}^5$.
	
	\subsection{Задача 3}
	
	
	\subsection{Задача 4}
	
	$Q(x_1,x_2,x_3) = 4 \cdot x_1^2 - 3 \cdot x_3^2 - 4 \cdot x_1x_2 + 12 \cdot x_1x_3 + 2 \cdot x_2x_3 = \bigl[4x_1^2 +4x_1(3x_3 - x_2) + (3x_3 - x_2)^2 \bigr] - (3x_3 - x_2)^2 - 3 \cdot x_3^2 + 2 \cdot x_2x_3$. \\
	
	$x = S_1 \cdot y$.
	\[ S_1 =
	\begin{pmatrix}
	0.5 & 0.5 & -1.5  \\
	0 & 1 & 0 \\
	0 & 0 & 1 \\
	\end{pmatrix}
	\]
	
	
	
	Пусть $y_1 = 2x_1 + 3x_3 - x_2$, $y_2 = x_2$, $y_3 = x_3$ \\
	$Q = y_1^2 -9y_3^2 + 6y_2y_3 - y_2^2 - 3 \cdot y_3^2 + 2 \cdot y_2y_3 = y_1^2 -12y_3^2 + 8y_2y_3 - y_2^2 = y_1^2 -\bigl[ y_2^2 - 8y_2y_3 + 16y_3^2 \bigr] + 4y_3^2$.\\
	
	$y = S_2 \cdot z$. 
	
	\[ S_2 =
	\begin{pmatrix}
	1 & 0 & 0 \\
	0 & 1 & 4 \\
	0 & 0 & 1 \\
	\end{pmatrix}
	\]
	
	Пусть $z_1 = y_1$, $z_2 = y_2 - 4y_3$, $z_3 = y_3$. \\
	$Q = z_1^2 - z_2^2 + 4z_3^2$.
	
	Матрица данной квадратичной формы -- это	\[ A =
	\begin{pmatrix}
	4 & -2 & 6 \\
	-2 & 0 & 1 \\
	6 & 1 & -3 \\
	\end{pmatrix}
	\]
	
	Матрица преобразований $S$ равна $S_1 \cdot S_2$:
	\[ S =
	\begin{pmatrix}
	1 & 1 & 1 \\
	0 & 1 & 4 \\
	0 & 0 & 1 \\
	\end{pmatrix}
	\]
	
	Тогда проверим, что найденная матрица канонической формы, которая получается равна diag(1,-1,4) также равна равна $S^T \cdot A \cdot S$. Это действительно так. (Проверил с помощью программы, код приложил к письму), а значит $S$ и есть матрица искомой замены. 
	

	\end{document}