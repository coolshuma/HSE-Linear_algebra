\documentclass[a4paper,12pt]{article}
\usepackage{amsmath}
\usepackage{cmap}					% поиск в PDF
\usepackage{mathtext} 				% русские буквы в формулах
\usepackage[T2A]{fontenc}			% кодировка
\usepackage[utf8]{inputenc}			% кодировка исходного текста
\usepackage[english,russian]{babel}	% локализация и переносы
\usepackage{amssymb}				    % Для красивого (!) \mathbb с  буквами и цифрами
\usepackage{mathbbol}

% Изменим формат \section и \subsection:
\usepackage{titlesec}
\titleformat{\section}
{\vspace{1cm}\centering\LARGE\bfseries}	% Стиль заголовка
{}										% префикс
{0pt}									% Расстояние между префиксом и заголовком
{} 										% Как отображается префикс
\titleformat{\subsection}				% Аналогично для \subsection
{\Large\bfseries}
{}
{0pt}
{}

%% Отступы между абзацами и в начале абзаца 
\setlength{\parindent}{0pt}
\setlength{\parskip}{\medskipamount}
\begin{document}
	\section{Индивидульное домашнее задание 3 \\ Шумилкин Андрей. Группа 163 \\ Вариант 48	. } 
	\subsection{Задача 1} 
	
	а) Составим из данных векторов матрицу и найдем ее ранг -- он и будет совпадать с рангом системы, то есть с размерностью базиса образованного этими векторами. А в базис могут войти вектора, у которых в столбце есть базисная переменная, при этом только по одному с каждой <<ступеньки>>.
	
	\[
	\begin{pmatrix}
	-13 & -34 & -35 & 9 \\
	3 & 4 & 5 & 2 \\
	-3 & 1 & 5 & 2 \\
	35 & 21 & -9 & -22 \\
	5 & 4 & -1 & -5 \\
	\end{pmatrix}
	\]
	
	Добавим к 1-ой строке 2-ую, домноженную на 4:
	
	\[
	\begin{pmatrix}
	-1 & -18 & -15 & 17 \\
	3 & 4 & 5 & 2 \\
	-3 & 1 & 5 & 2 \\
	35 & 21 & -9 & -22 \\
	5 & 4 & -1 & -5 \\
	\end{pmatrix}
	\]
	
	Домножим 1-ую строку на -1:
	
	\[
	\begin{pmatrix}
	1 & 18 & 15 & -17 \\
	3 & 4 & 5 & 2 \\
	-3 & 1 & 5 & 2 \\
	35 & 21 & -9 & -22 \\
	5 & 4 & -1 & -5 \\
	\end{pmatrix}
	\]
	
	Добавим к 2-ой строке 1-ую, домноженную на -3:
	
	\[
	\begin{pmatrix}
	1 & 18 & 15 & -17 \\
	0 & -50 & -40 & 53 \\
	-3 & 1 & 5 & 2 \\
	35 & 21 & -9 & -22 \\
	5 & 4 & -1 & -5 \\
	\end{pmatrix}
	\]
	
	Добавим к 3-ой строке 1-ую, домноженную на 3:
	
	\[
	\begin{pmatrix}
	1 & 18 & 15 & -17 \\
	0 & -50 & -40 & 53 \\
	0 & 55 & 50 & -49 \\
	35 & 21 & -9 & -22 \\
	5 & 4 & -1 & -5 \\
	\end{pmatrix}
	\]
	
	Добавим к 4-ой строке 1-ую, домноженную на -35:
	
	\[
	\begin{pmatrix}
	1 & 18 & 15 & -17 \\
	0 & -50 & -40 & 53 \\
	0 & 55 & 50 & -49 \\
	0 & -609 & -534 & 573 \\
	5 & 4 & -1 & -5 \\
	\end{pmatrix}
	\]
	
	Добавим к 5-ой строке 1-ую, домноженную на -5:
	
	\[
	\begin{pmatrix}
	1 & 18 & 15 & -17 \\
	0 & -50 & -40 & 53 \\
	0 & 55 & 50 & -49 \\
	0 & -609 & -534 & 573 \\
	0 & -86 & -76 & 80 \\
	\end{pmatrix}
	\]
	
	Разделим 4-ую строку на 3:
	
	\[
	\begin{pmatrix}
	1 & 18 & 15 & -17 \\
	0 & -50 & -40 & 53 \\
	0 & 55 & 50 & -49 \\
	0 & -203 & -178 & 191 \\
	0 & -86 & -76 & 80 \\
	\end{pmatrix}
	\]
	
	Добавим к 4-ой строке 2-ую, домноженную на -4:
	
	\[
	\begin{pmatrix}
	1 & 18 & 15 & -17 \\
	0 & -50 & -40 & 53 \\
	0 & 55 & 50 & -49 \\
	0 & -3 & -18 & -21 \\
	0 & -86 & -76 & 80 \\
	\end{pmatrix}
	\]
	
	Разделим 4-ую строку на -3:
	
	\[
	\begin{pmatrix}
	1 & 18 & 15 & -17 \\
	0 & -50 & -40 & 53 \\
	0 & 55 & 50 & -49 \\
	0 & 1 & 6 & 7 \\
	0 & -86 & -76 & 80 \\
	\end{pmatrix}
	\]
	
	Поменяем местами 4-ую и 2-ую строку:
	
	\[
	\begin{pmatrix}
	1 & 18 & 15 & -17 \\
	0 & 1 & 6 & 7 \\
	0 & 55 & 50 & -49 \\
	0 & -50 & -40 & 53 \\
	0 & -86 & -76 & 80 \\
	\end{pmatrix}
	\]
	
	Добавим к 3-ой строке 2-ую, домноженную на -55:
	
	\[
	\begin{pmatrix}
	1 & 18 & 15 & -17 \\
	0 & 1 & 6 & 7 \\
	0 & 0 & -280 & -434 \\
	0 & -50 & -40 & 53 \\
	0 & -86 & -76 & 80 \\
	\end{pmatrix}
	\]
	
	Добавим к 4-ой строке 2-ую, домноженную на 50:
	
	\[
	\begin{pmatrix}
	1 & 18 & 15 & -17 \\
	0 & 1 & 6 & 7 \\
	0 & 0 & -280 & -434 \\
	0 & 0 & 260 & 403 \\
	0 & -86 & -76 & 80 \\
	\end{pmatrix}
	\]
	
	Добавим к 5-ой строке 2-ую, домноженную на 86:
	
	\[
	\begin{pmatrix}
	1 & 18 & 15 & -17 \\
	0 & 1 & 6 & 7 \\
	0 & 0 & -280 & -434 \\
	0 & 0 & 260 & 403 \\
	0 & 0 & 440 & 682 \\
	\end{pmatrix}
	\]
	
	Добавим к 3-ой строке 4-ую, домноженную на 1:
	
	\[
	\begin{pmatrix}
	1 & 18 & 15 & -17 \\
	0 & 1 & 6 & 7 \\
	0 & 0 & -20 & -31 \\
	0 & 0 & 260 & 403 \\
	0 & 0 & 440 & 682 \\
	\end{pmatrix}
	\]
	
	Домножим 3-ую строку на -1:
	
	\[
	\begin{pmatrix}
	1 & 18 & 15 & -17 \\
	0 & 1 & 6 & 7 \\
	0 & 0 & 20 & 31 \\
	0 & 0 & 260 & 403 \\
	0 & 0 & 440 & 682 \\
	\end{pmatrix}
	\]
	
	Добавим к 4-ой строке 3-ую, домноженную на -13:
	
	\[
	\begin{pmatrix}
	1 & 18 & 15 & -17 \\
	0 & 1 & 6 & 7 \\
	0 & 0 & 20 & 31 \\
	0 & 0 & 0 & 0 \\
	0 & 0 & 440 & 682 \\
	\end{pmatrix}
	\]
	
	Добавим к 5-ой строке 3-ую, домноженную на -22:
	
	\[
	\begin{pmatrix}
	1 & 18 & 15 & -17 \\
	0 & 1 & 6 & 7 \\
	0 & 0 & 20 & 31 \\
	0 & 0 & 0 & 0 \\
	0 & 0 & 0 & 0 \\
	\end{pmatrix}
	\]
	
	Базисом могут являться векторы $u_1, u_2$ и $u_3$.\\
	
	б) Базис в пространстве $\mathbb{R}^5$ должен иметь размерность 5. Приведем наши вектора к ступенчатому виду и посмотрим какими векторами мы можем это сделать.
	
	\[
	\begin{pmatrix}
	-13 & -34 & -35 \\
	3 & 4 & 5 \\
	-3 & 1 & 5 \\
	35 & 21 & -9 \\
	5 & 4 & -1 \\
	\end{pmatrix}
	\]
	
	Добавим к 1-ой строке 2-ую, домноженную на 4:
	
	\[
	\begin{pmatrix}
	-1 & -18 & -15 \\
	3 & 4 & 5 \\
	-3 & 1 & 5 \\
	35 & 21 & -9 \\
	5 & 4 & -1 \\
	\end{pmatrix}
	\]
	
	Домножим 1-ую строку на -1:
	
	\[
	\begin{pmatrix}
	1 & 18 & 15 \\
	3 & 4 & 5 \\
	-3 & 1 & 5 \\
	35 & 21 & -9 \\
	5 & 4 & -1 \\
	\end{pmatrix}
	\]
	
	Добавим к 2-ой строке 1-ую, домноженную на -3:
	
	\[
	\begin{pmatrix}
	1 & 18 & 15 \\
	0 & -50 & -40 \\
	-3 & 1 & 5 \\
	35 & 21 & -9 \\
	5 & 4 & -1 \\
	\end{pmatrix}
	\]
	
	Добавим к 3-ой строке 1-ую, домноженную на 3:
	
	\[
	\begin{pmatrix}
	1 & 18 & 15 \\
	0 & -50 & -40 \\
	0 & 55 & 50 \\
	35 & 21 & -9 \\
	5 & 4 & -1 \\
	\end{pmatrix}
	\]
	
	Добавим к 4-ой строке 1-ую, домноженную на -35:
	
	\[
	\begin{pmatrix}
	1 & 18 & 15 \\
	0 & -50 & -40 \\
	0 & 55 & 50 \\
	0 & -609 & -534 \\
	5 & 4 & -1 \\
	\end{pmatrix}
	\]
	
	Добавим к 5-ой строке 1-ую, домноженную на -5:
	
	\[
	\begin{pmatrix}
	1 & 18 & 15 \\
	0 & -50 & -40 \\
	0 & 55 & 50 \\
	0 & -609 & -534 \\
	0 & -86 & -76 \\
	\end{pmatrix}
	\]
	
	Разделим 2-ую строку на 5:
	
	\[
	\begin{pmatrix}
	1 & 18 & 15 \\
	0 & -10 & -8 \\
	0 & 55 & 50 \\
	0 & -609 & -534 \\
	0 & -86 & -76 \\
	\end{pmatrix}
	\]
	
	Разделим 3-ую строку на 5:
	
	\[
	\begin{pmatrix}
	1 & 18 & 15 \\
	0 & -10 & -8 \\
	0 & 11 & 10 \\
	0 & -609 & -534 \\
	0 & -86 & -76 \\
	\end{pmatrix}
	\]
	
	Добавим к 2-ой строке 3-ую, домноженную на 1:
	
	\[
	\begin{pmatrix}
	1 & 18 & 15 \\
	0 & 1 & 2 \\
	0 & 11 & 10 \\
	0 & -609 & -534 \\
	0 & -86 & -76 \\
	\end{pmatrix}
	\]
	
	Добавим к 3-ой строке 2-ую, домноженную на -11:
	
	\[
	\begin{pmatrix}
	1 & 18 & 15 \\
	0 & 1 & 2 \\
	0 & 0 & -12 \\
	0 & -609 & -534 \\
	0 & -86 & -76 \\
	\end{pmatrix}
	\]
	
	Добавим к 4-ой строке 2-ую, домноженную на 609:
	
	\[
	\begin{pmatrix}
	1 & 18 & 15 \\
	0 & 1 & 2 \\
	0 & 0 & -12 \\
	0 & 0 & 684 \\
	0 & -86 & -76 \\
	\end{pmatrix}
	\]
	
	Добавим к 5-ой строке 2-ую, домноженную на 86:
	
	\[
	\begin{pmatrix}
	1 & 18 & 15 \\
	0 & 1 & 2 \\
	0 & 0 & -12 \\
	0 & 0 & 684 \\
	0 & 0 & 96 \\
	\end{pmatrix}
	\]
	
	Добавим к 4-ой строке 3-ую, домноженную на 57:
	
	\[
	\begin{pmatrix}
	1 & 18 & 15 \\
	0 & 1 & 2 \\
	0 & 0 & -12 \\
	0 & 0 & 0 \\
	0 & 0 & 96 \\
	\end{pmatrix}
	\]
	
	Добавим к 5-ой строке 3-ую, домноженную на 8:
	
	\[
	\begin{pmatrix}
	1 & 18 & 15 \\
	0 & 1 & 2 \\
	0 & 0 & -12 \\
	0 & 0 & 0 \\
	0 & 0 & 0 \\
	\end{pmatrix}
	\]
	
	Видим, что на четвертом и пятом <<уровне>> не хватает базисных переменных, при этом во время преобразований мы не прибавляли четвертую и пятую строку ни к каким другим. Мы можем дополнить наш набор векторов до базиса данного пространства $\vec b_1 = (0,\ 0,\ 0,\ 1,\ 0),\ \vec b_2 = (0,\ 0,\ 0,\ 0,\ 1)$, поскольку во время приведения к ступенчатому виду мы только прибавляли к четвертой и пятой строке, а не эти строки к чему-либо, а при записи векторов $b_1$ и $b_2$ у них в первых трех строках стоят нули, а значит их прибавление к четвертой и пятой строке ничего не изменит и при приведении к ступенчатому виду матрица, составленная из векторов $(u_1, u_2, u_3, b_1, b_2)$ так же будет иметь ранг 5, а значит 
	эти вектора будут являться базисом в $\mathbb{R}^5$.
	
	
	
	\subsection{Задача 2} 
	
	Будем рассматривать искомую ОСЛУ так же, как и линейную оболочку данных векторов -- то есть как описание подпространства $U$. \\
	Тогда нам нужно просто перейти от одного способо задания подпространства(с помощью линейной оболочки) к другому(с помощью СЛУ).
	
	\[
	\begin{pmatrix}
	-12 & 24 & 29 & 5 & | & 1 & 0 & 0 & 0 \\
	11 & 3 & -4 & -1 & | & 0 & 1 & 0 & 0 \\
	-68 & 36 & 74 & 14 & | & 0 & 0 & 1 & 0 \\
	-38 & 126 & 137 & 23 & | & 0 & 0 & 0 & 1 \\
	\end{pmatrix}
	\]
	
	Добавим к 1-ой строке 2-ую, домноженную на 1:
	
	\[
	\begin{pmatrix}
	-1 & 27 & 25 & 4 & | & 1 & 1 & 0 & 0 \\
	11 & 3 & -4 & -1 & | & 0 & 1 & 0 & 0 \\
	-68 & 36 & 74 & 14 & | & 0 & 0 & 1 & 0 \\
	-38 & 126 & 137 & 23 & | & 0 & 0 & 0 & 1 \\
	\end{pmatrix}
	\]
	
	Домножим 1-ую строку на -1:
	
	\[
	\begin{pmatrix}
	1 & -27 & -25 & -4 & | & -1 & -1 & 0 & 0 \\
	11 & 3 & -4 & -1 & | & 0 & 1 & 0 & 0 \\
	-68 & 36 & 74 & 14 & | & 0 & 0 & 1 & 0 \\
	-38 & 126 & 137 & 23 & | & 0 & 0 & 0 & 1 \\
	\end{pmatrix}
	\]
	
	Добавим к 2-ой строке 1-ую, домноженную на -11:
	
	\[
	\begin{pmatrix}
	1 & -27 & -25 & -4 & | & -1 & -1 & 0 & 0 \\
	0 & 300 & 271 & 43 & | & 11 & 12 & 0 & 0 \\
	-68 & 36 & 74 & 14 & | & 0 & 0 & 1 & 0 \\
	-38 & 126 & 137 & 23 & | & 0 & 0 & 0 & 1 \\
	\end{pmatrix}
	\]
	
	Добавим к 3-ой строке 1-ую, домноженную на 68:
	
	\[
	\begin{pmatrix}
	1 & -27 & -25 & -4 & | & -1 & -1 & 0 & 0 \\
	0 & 300 & 271 & 43 & | & 11 & 12 & 0 & 0 \\
	0 & -1800 & -1626 & -258 & | & -68 & -68 & 1 & 0 \\
	-38 & 126 & 137 & 23 & | & 0 & 0 & 0 & 1 \\
	\end{pmatrix}
	\]
	
	Добавим к 4-ой строке 1-ую, домноженную на 38:
	
	\[
	\begin{pmatrix}
	1 & -27 & -25 & -4 & | & -1 & -1 & 0 & 0 \\
	0 & 300 & 271 & 43 & | & 11 & 12 & 0 & 0 \\
	0 & -1800 & -1626 & -258 & | & -68 & -68 & 1 & 0 \\
	0 & -900 & -813 & -129 & | & -38 & -38 & 0 & 1 \\
	\end{pmatrix}
	\]
	
	Добавим к 3-ой строке 2-ую, домноженную на 6:
	
	\[
	\begin{pmatrix}
	1 & -27 & -25 & -4 & | & -1 & -1 & 0 & 0 \\
	0 & 300 & 271 & 43 & | & 11 & 12 & 0 & 0 \\
	0 & 0 & 0 & 0 & | & -2 & 4 & 1 & 0 \\
	0 & -900 & -813 & -129 & | & -38 & -38 & 0 & 1 \\
	\end{pmatrix}
	\]
	
	Добавим к 4-ой строке 2-ую, домноженную на 3:
	
	\[
	\begin{pmatrix}
	1 & -27 & -25 & -4 & | & -1 & -1 & 0 & 0 \\
	0 & 300 & 271 & 43 & | & 11 & 12 & 0 & 0 \\
	0 & 0 & 0 & 0 & | & -2 & 4 & 1 & 0 \\
	0 & 0 & 0 & 0 & | & -5 & -2 & 0 & 1 \\
	\end{pmatrix}
	\]
	
	Добавим к 2-ому столбцу 4-ый, домноженный на -7:
	
	\[
	\begin{pmatrix}
	1 & 1 & -25 & -4 & | & -1 & -1 & 0 & 0 \\
	0 & -1 & 271 & 43 & | & 11 & 12 & 0 & 0 \\
	0 & 0 & 0 & 0 & | & -2 & 4 & 1 & 0 \\
	0 & 0 & 0 & 0 & | & -5 & -2 & 0 & 1 \\
	\end{pmatrix}
	\]
	
	Добавим к 1-ой строке 2-ую, домноженную на 1:
	
	\[
	\begin{pmatrix}
	1 & 0 & 246 & 39 & | & 10 & 11 & 0 & 0 \\
	0 & -1 & 271 & 43 & | & 11 & 12 & 0 & 0 \\
	0 & 0 & 0 & 0 & | & -2 & 4 & 1 & 0 \\
	0 & 0 & 0 & 0 & | & -5 & -2 & 0 & 1 \\
	\end{pmatrix}
	\]
	
	Домножим 2-ую строку на -1:
	
	\[
	\begin{pmatrix}
	1 & 0 & 246 & 39 & | & 10 & 11 & 0 & 0 \\
	0 & 1 & -271 & -43 & | & -11 & -12 & 0 & 0 \\
	0 & 0 & 0 & 0 & | & -2 & 4 & 1 & 0 \\
	0 & 0 & 0 & 0 & | & -5 & -2 & 0 & 1 \\
	\end{pmatrix}
	\]
	
	Добавим к 3-ому столбцу 1-ый, домноженный на -246:
	
	\[
	\begin{pmatrix}
	1 & 0 & 0 & 39 & | & 10 & 11 & 0 & 0 \\
	0 & 1 & -271 & -43 & | & -11 & -12 & 0 & 0 \\
	0 & 0 & 0 & 0 & | & -2 & 4 & 1 & 0 \\
	0 & 0 & 0 & 0 & | & -5 & -2 & 0 & 1 \\
	\end{pmatrix}
	\]
	
	Добавим к 4-ому столбцу 1-ый, домноженный на -39:
	
	\[
	\begin{pmatrix}
	1 & 0 & 0 & 0 & | & 10 & 11 & 0 & 0 \\
	0 & 1 & -271 & -43 & | & -11 & -12 & 0 & 0 \\
	0 & 0 & 0 & 0 & | & -2 & 4 & 1 & 0 \\
	0 & 0 & 0 & 0 & | & -5 & -2 & 0 & 1 \\
	\end{pmatrix}
	\]
	
	Добавим к 3-ому столбцу 2-ый, домноженный на 271:
	
	\[
	\begin{pmatrix}
	1 & 0 & 0 & 0 & | & 10 & 11 & 0 & 0 \\
	0 & 1 & 0 & -43 & | & -11 & -12 & 0 & 0 \\
	0 & 0 & 0 & 0 & | & -2 & 4 & 1 & 0 \\
	0 & 0 & 0 & 0 & | & -5 & -2 & 0 & 1 \\
	\end{pmatrix}
	\]
	
	Добавим к 4-ому столбцу 2-ый, домноженный на 43:
	
	\[
	\begin{pmatrix}
	1 & 0 & 0 & 0 & | & 10 & 11 & 0 & 0 \\
	0 & 1 & 0 & 0 & | & -11 & -12 & 0 & 0 \\
	0 & 0 & 0 & 0 & | & -2 & 4 & 1 & 0 \\
	0 & 0 & 0 & 0 & | & -5 & -2 & 0 & 1 \\
	\end{pmatrix}
	\]
	
	\[ S=
	\begin{pmatrix}
	10 & 11 & 0 & 0 \\
	-11 & -12 & 0 & 0 \\
	-2 & 4 & 1 & 0 \\
	-5 & -2 & 0 & 1 \\
	\end{pmatrix}
	\]
	
	rg A = 2. \\
	
	Тогда матрица искомой системы:
	
	\[ \Psi=
	\begin{pmatrix}
	-2 & 4 & 1 & 0 \\
	-5 & -2 & 0 & 1 \\
	\end{pmatrix}
	\]
	
	 И получаем следующую ОСЛУ: 
	 \[
		 \begin{cases}
		 -2x_1 + 4x_2 + x_3 = 0 \\
		 -5x_1 -2x_2 + x_4 = 0
		 \end{cases}
	 \]
	
	
	
	\subsection{Задача 3} 
	$\vec a_1 = (-4,\ -1,\ -6,\ 0),\\ \vec a_2 = (0,\ -3,\ -2,\ -2),\\ 
	\vec a_3 = (-8,\ 1,\ -10,\ 2),\\ \vec a_4 = (-4,\ 5,\ -2,\ 4)$. 
	
	$L_1 = <\vec a_1,\ \vec a_2,\ \vec a_3,\ \vec a_4>$.
	
	$\vec b_1 = (8,\ 2,\ 12,\ 0),\\ \vec b_2 = (8,\ -1,\ 10,\ -2),\\ 
	\vec b_3 = (8,\ 5,\ 14,\ 2),\\ \vec b_4 = (-8,\ 4,\ -8,\ 4)$. 
	
	$L_2 = <\vec b_1,\ \vec b_2,\ \vec b_3,\ \vec b_4>$.
	
	Найдем базис $L_1$, составив матрицу из векторов, линейной оболочкой которых она является и с помощью линейных преобразований поймем какие векторы являются линейно-независимыми:
	
	\[
	\begin{pmatrix}
	-4 & -1 & -6 & 0 \\
	0 & -3 & -2 & -2 \\
	-8 & 1 & -10 & 2 \\
	-4 & 5 & -2 & 4 \\
	\end{pmatrix}
	\]
	
	Добавим к 3-ой строке 1-ую, домноженную на -2:
	
	\[
	\begin{pmatrix}
	-4 & -1 & -6 & 0 \\
	0 & -3 & -2 & -2 \\
	0 & 3 & 2 & 2 \\
	-4 & 5 & -2 & 4 \\
	\end{pmatrix}
	\]
	
	Добавим к 4-ой строке 1-ую, домноженную на -1:
	
	\[
	\begin{pmatrix}
	-4 & -1 & -6 & 0 \\
	0 & -3 & -2 & -2 \\
	0 & 3 & 2 & 2 \\
	0 & 6 & 4 & 4 \\
	\end{pmatrix}
	\]
	
	Добавим к 3-ой строке 2-ую, домноженную на 1:
	
	\[
	\begin{pmatrix}
	-4 & -1 & -6 & 0 \\
	0 & -3 & -2 & -2 \\
	0 & 0 & 0 & 0 \\
	0 & 6 & 4 & 4 \\
	\end{pmatrix}
	\]
	
	Добавим к 4-ой строке 2-ую, домноженную на 2:
	
	\[
	\begin{pmatrix}
	-4 & -1 & -6 & 0 \\
	0 & -3 & -2 & -2 \\
	0 & 0 & 0 & 0 \\
	0 & 0 & 0 & 0 \\
	\end{pmatrix}
	\]
	
	Как видим, базисом $L_1$ являются $\vec a_1,\ \vec a_2$ и тогда dim $L_1 = 2$. \\
	
	Теперь таким же образом найдем базис $L_2$:
	
	\[
	\begin{pmatrix}
	8 & 2 & 12 & 0 \\
	8 & -1 & 10 & -2 \\
	8 & 5 & 14 & 2 \\
	-8 & 4 & -8 & 4 \\
	\end{pmatrix}
	\]
	
	Добавим к 2-ой строке 1-ую, домноженную на -1:
	
	\[
	\begin{pmatrix}
	8 & 2 & 12 & 0 \\
	0 & -3 & -2 & -2 \\
	8 & 5 & 14 & 2 \\
	-8 & 4 & -8 & 4 \\
	\end{pmatrix}
	\]
	
	Добавим к 3-ой строке 1-ую, домноженную на -1:
	
	\[
	\begin{pmatrix}
	8 & 2 & 12 & 0 \\
	0 & -3 & -2 & -2 \\
	0 & 3 & 2 & 2 \\
	-8 & 4 & -8 & 4 \\
	\end{pmatrix}
	\]
	
	Добавим к 4-ой строке 1-ую, домноженную на 1:
	
	\[
	\begin{pmatrix}
	8 & 2 & 12 & 0 \\
	0 & -3 & -2 & -2 \\
	0 & 3 & 2 & 2 \\
	0 & 6 & 4 & 4 \\
	\end{pmatrix}
	\]
	
	Добавим к 3-ой строке 2-ую, домноженную на 1:
	
	\[
	\begin{pmatrix}
	8 & 2 & 12 & 0 \\
	0 & -3 & -2 & -2 \\
	0 & 0 & 0 & 0 \\
	0 & 6 & 4 & 4 \\
	\end{pmatrix}
	\]
	
	Добавим к 4-ой строке 2-ую, домноженную на 2:
	
	\[
	\begin{pmatrix}
	8 & 2 & 12 & 0 \\
	0 & -3 & -2 & -2 \\
	0 & 0 & 0 & 0 \\
	0 & 0 & 0 & 0 \\
	\end{pmatrix}
	\]
	
	Как видим, базисом $L_2$ являются $\vec b_1,\ \vec b_2$ и тогда dim $L_2 = 2$. \\
	
	Теперь найдем базис $U = L_1 + L_2$, для чего выпишем базисные векторы для $L_1$ и $L_2$ в матрицу и с помощью линейных преобразований найдем линейно-независимые векторы: \\
	
	\[
	\begin{pmatrix}
	-4 & -1 & -6 & 0 \\
	0 & -3 & -2 & -2 \\
	8 & 2 & 12 & 0 \\
	8 & -1 & 10 & -2 \\
	\end{pmatrix}
	\]
	
	Добавим к 3-ой строке 1-ую, домноженную на 2:
	
	\[
	\begin{pmatrix}
	-4 & -1 & -6 & 0 \\
	0 & -3 & -2 & -2 \\
	0 & 0 & 0 & 0 \\
	8 & -1 & 10 & -2 \\
	\end{pmatrix}
	\]
	
	Добавим к 4-ой строке 1-ую, домноженную на 2:
	
	\[
	\begin{pmatrix}
	-4 & -1 & -6 & 0 \\
	0 & -3 & -2 & -2 \\
	0 & 0 & 0 & 0 \\
	0 & -3 & -2 & -2 \\
	\end{pmatrix}
	\]
	
	Добавим к 4-ой строке 2-ую, домноженную на -1:
	
	\[
	\begin{pmatrix}
	-4 & -1 & -6 & 0 \\
	0 & -3 & -2 & -2 \\
	0 & 0 & 0 & 0 \\
	0 & 0 & 0 & 0 \\
	\end{pmatrix}
	\]
	
	Таким образом базисом $U$ будут $\vec a_1, \vec a_2$ и dim $U = 2$. \\
	
	Вектор, который входит в пересечение должен раскладываться по базису каждого из подпространств, т.е. отсюда $\alpha_1 \cdot a_1 + \alpha_2 \cdot a_2 = \beta_1 \cdot b_1 + \beta_2 \cdot b_2$. $\alpha_1 \cdot a_1 + \alpha_2 \cdot a_2 - \beta_1 \cdot b_1 - \beta_2 \cdot b_2$
	Тогда составим и решим с помощью метода Гаусса СЛУ для получившегося уравнения: 
	
	\[
	\begin{pmatrix}
	-4 & 0 & -8 & -8 \\
	-1 & -3 & -2 & 1 \\
	-6 & -2 & -12 & -10 \\
	0 & -2 & 0 & 2 \\
	\end{pmatrix}
	\]
	
	Добавим к 1-ой строке 2-ую, домноженную на -5:
	
	\[
	\begin{pmatrix}
	1 & 15 & 2 & -13 \\
	-1 & -3 & -2 & 1 \\
	-6 & -2 & -12 & -10 \\
	0 & -2 & 0 & 2 \\
	\end{pmatrix}
	\]
	
	Добавим к 2-ой строке 1-ую, домноженную на 1:
	
	\[
	\begin{pmatrix}
	1 & 15 & 2 & -13 \\
	0 & 12 & 0 & -12 \\
	-6 & -2 & -12 & -10 \\
	0 & -2 & 0 & 2 \\
	\end{pmatrix}
	\]
	
	Добавим к 3-ой строке 1-ую, домноженную на 6:
	
	\[
	\begin{pmatrix}
	1 & 15 & 2 & -13 \\
	0 & 12 & 0 & -12 \\
	0 & 88 & 0 & -88 \\
	0 & -2 & 0 & 2 \\
	\end{pmatrix}
	\]
	
	Разделим 2-ую строку на 12:
	
	\[
	\begin{pmatrix}
	1 & 15 & 2 & -13 \\
	0 & 1 & 0 & -1 \\
	0 & 88 & 0 & -88 \\
	0 & -2 & 0 & 2 \\
	\end{pmatrix}
	\]
	
	Добавим к 3-ой строке 2-ую, домноженную на -88:
	
	\[
	\begin{pmatrix}
	1 & 15 & 2 & -13 \\
	0 & 1 & 0 & -1 \\
	0 & 0 & 0 & 0 \\
	0 & -2 & 0 & 2 \\
	\end{pmatrix}
	\]
	
	Добавим к 4-ой строке 2-ую, домноженную на 2:
	
	\[
	\begin{pmatrix}
	1 & 15 & 2 & -13 \\
	0 & 1 & 0 & -1 \\
	0 & 0 & 0 & 0 \\
	0 & 0 & 0 & 0 \\
	\end{pmatrix}
	\]
	
	Добавим к 1-ой строке 2-ую, домноженную на -15:
	
	\[
	\begin{pmatrix}
	1 & 0 & 2 & 2 \\
	0 & 1 & 0 & -1 \\
	0 & 0 & 0 & 0 \\
	0 & 0 & 0 & 0 \\
	\end{pmatrix}
	\]
	
	Получаем:
	\[
	\begin{cases}
	a_1 = -2b_1 -2 b_2 \\
	a_2 = b_2 
	\end{cases}
	\]
	
	Составим ФСР, в которой будет два вектора, так как ранг матрицы равен двум, взяв значения (1,0) и (1, -1) для $b_1, b_2$ соответственно. 
	
	Получим:
	\[ X_1 = 
	\begin{pmatrix}
    -2 \\
	0  \\
	1 \\
	0 \\
	\end{pmatrix},\ 
	X_2 = 
	\begin{pmatrix}
	0 \\
	-1 \\
	1 \\
	-1 \\
	\end{pmatrix}
	\]
	
	Мы знаем, что сумма размерностей суммы и пересечения подпространств равна сумме размерностей исходных подпространств $\Rightarrow$ dim $V$ = 2.  \\
	Пусть базисом $V$ будут $\vec z_1, \vec z_2$. Найдем данные вектора, выразив через базис подпространства $L_1$ c помощью найденных в ФСР значений: 
	\[
		z_1 = -2 \cdot a_1 = (8,\ 2,\ 12,\ 0) 
	\]
	\[
		z_2 = -a_2 = (0,\ 3,\ 2,\ 2) 
	\]
	
	\textbf{Ответ:\\} Базисом $L_1$ являются $\vec a_1,\ \vec a_2$ и тогда dim $L_1 = 2$. \\
	Базисом $L_2$ являются $\vec b_1,\ \vec b_2$ и тогда dim $L_2 = 2$. \\
	Базисом $U$ будут $\vec a_1, \vec a_2$ и dim $U = 2$.\\
	Базисом $V$ будут $\vec z_1, \vec z_2$ и dim $V = 2$.
	
	
	\subsection{Задача 4} 
	а) Для того, чтобы $\mathbb{R}^4$ разлогалось в прямую сумму данных подпространств необходимо, чтобы размерность их преесечения была равна нулю и чтобы размерность их суммы была равна четырем. \\
	Сразу заметим, что размерность исходных подпространств равна 2 и 2, поскольку для первого:
	\[
	\begin{pmatrix}
	-5 & 5 & -8 & 8 \\
	-6 & 20 & 7 & 5 \\
	\end{pmatrix}
	\]
	
	Добавим к 1-ой строке 2-ую, домноженную на -1:
	
	\[
	\begin{pmatrix}
	1 & -15 & -15 & 3 \\
	-6 & 20 & 7 & 5 \\
	\end{pmatrix}
	\]
	
	Добавим к 2-ой строке 1-ую, домноженную на 6:
	
	\[
	\begin{pmatrix}
	1 & -15 & -15 & 3 \\
	0 & -70 & -83 & 23 \\
	\end{pmatrix}
	\]
	
	Для второй: \\
	\[
	\begin{pmatrix}
	9 & -10 & -13 & 0 \\
	-12 & 2 & -4 & 1 \\
	\end{pmatrix}
	\]
	
	Добавим к 2-ой строке 1-ую, домноженную на 1:
	
	\[
	\begin{pmatrix}
	9 & -10 & -13 & 0 \\
	-3 & -8 & -17 & 1 \\
	\end{pmatrix}
	\]
	
	Добавим к 1-ой строке 2-ую, домноженную на 3:
	
	\[
	\begin{pmatrix}
	0 & -34 & -64 & 3 \\
	-3 & -8 & -17 & 1 \\
	\end{pmatrix}
	\]
	
	Поменяем местами 1-ую и 2-ую строку:
	
	\[
	\begin{pmatrix}
	-3 & -8 & -17 & 1 \\
	0 & -34 & -64 & 3 \\
	\end{pmatrix}
	\]
	
	
	Далее найдем размерность их суммы, записав в матрицу и найдя ее ранг: \\
	\[
	\begin{pmatrix}
	-5 & 5 & -8 & 8 \\
	-6 & 20 & 7 & 5 \\
	9 & -10 & -13 & 0 \\
	-12 & 2 & -4 & 1 \\
	\end{pmatrix}
	\]
	
	Добавим к 3-ой строке 2-ую, домноженную на 1:
	
	\[
	\begin{pmatrix}
	-5 & 5 & -8 & 8 \\
	-6 & 20 & 7 & 5 \\
	3 & 10 & -6 & 5 \\
	-12 & 2 & -4 & 1 \\
	\end{pmatrix}
	\]
	
	Добавим к 4-ой строке 2-ую, домноженную на -2:
	
	\[
	\begin{pmatrix}
	-5 & 5 & -8 & 8 \\
	-6 & 20 & 7 & 5 \\
	3 & 10 & -6 & 5 \\
	0 & -38 & -18 & -9 \\
	\end{pmatrix}
	\]
	
	Добавим к 1-ой строке 2-ую, домноженную на -1:
	
	\[
	\begin{pmatrix}
	1 & -15 & -15 & 3 \\
	-6 & 20 & 7 & 5 \\
	3 & 10 & -6 & 5 \\
	0 & -38 & -18 & -9 \\
	\end{pmatrix}
	\]
	
	Добавим к 2-ой строке 3-ую, домноженную на 2:
	
	\[
	\begin{pmatrix}
	1 & -15 & -15 & 3 \\
	0 & 40 & -5 & 15 \\
	3 & 10 & -6 & 5 \\
	0 & -38 & -18 & -9 \\
	\end{pmatrix}
	\]
	
	Добавим к 3-ой строке 1-ую, домноженную на -3:
	
	\[
	\begin{pmatrix}
	1 & -15 & -15 & 3 \\
	0 & 40 & -5 & 15 \\
	0 & 55 & 39 & -4 \\
	0 & -38 & -18 & -9 \\
	\end{pmatrix}
	\]
	
	Разделим 2-ую строку на 5:
	
	\[
	\begin{pmatrix}
	1 & -15 & -15 & 3 \\
	0 & 8 & -1 & 3 \\
	0 & 55 & 39 & -4 \\
	0 & -38 & -18 & -9 \\
	\end{pmatrix}
	\]
	
	Добавим к 3-ой строке 4-ую, домноженную на 1:
	
	\[
	\begin{pmatrix}
	1 & -15 & -15 & 3 \\
	0 & 8 & -1 & 3 \\
	0 & 17 & 21 & -13 \\
	0 & -38 & -18 & -9 \\
	\end{pmatrix}
	\]
	
	Добавим к 4-ой строке 3-ую, домноженную на 2:
	
	\[
	\begin{pmatrix}
	1 & -15 & -15 & 3 \\
	0 & 8 & -1 & 3 \\
	0 & 17 & 21 & -13 \\
	0 & -4 & 24 & -35 \\
	\end{pmatrix}
	\]
	
	Добавим к 3-ой строке 4-ую, домноженную на 4:
	
	\[
	\begin{pmatrix}
	1 & -15 & -15 & 3 \\
	0 & 8 & -1 & 3 \\
	0 & 1 & 117 & -153 \\
	0 & -4 & 24 & -35 \\
	\end{pmatrix}
	\]
	
	Добавим к 2-ой строке 4-ую, домноженную на 2:
	
	\[
	\begin{pmatrix}
	1 & -15 & -15 & 3 \\
	0 & 0 & 47 & -67 \\
	0 & 1 & 117 & -153 \\
	0 & -4 & 24 & -35 \\
	\end{pmatrix}
	\]
	
	Добавим к 4-ой строке 3-ую, домноженную на 4:
	
	\[
	\begin{pmatrix}
	1 & -15 & -15 & 3 \\
	0 & 0 & 47 & -67 \\
	0 & 1 & 117 & -153 \\
	0 & 0 & 492 & -647 \\
	\end{pmatrix}
	\]
	
	Поменяем местами 2-ую и 3-ую строку:
	
	\[
	\begin{pmatrix}
	1 & -15 & -15 & 3 \\
	0 & 1 & 117 & -153 \\
	0 & 0 & 47 & -67 \\
	0 & 0 & 492 & -647 \\
	\end{pmatrix}
	\]
	
	Добавим к 4-ой строке 3-ую, домноженную на -10:
	
	\[
	\begin{pmatrix}
	1 & -15 & -15 & 3 \\
	0 & 1 & 117 & -153 \\
	0 & 0 & 47 & -67 \\
	0 & 0 & 22 & 23 \\
	\end{pmatrix}
	\]
	
	Добавим к 3-ой строке 4-ую, домноженную на -2:
	
	\[
	\begin{pmatrix}
	1 & -15 & -15 & 3 \\
	0 & 1 & 117 & -153 \\
	0 & 0 & 3 & -113 \\
	0 & 0 & 22 & 23 \\
	\end{pmatrix}
	\]
	
	Добавим к 4-ой строке 3-ую, домноженную на -7:
	
	\[
	\begin{pmatrix}
	1 & -15 & -15 & 3 \\
	0 & 1 & 117 & -153 \\
	0 & 0 & 3 & -113 \\
	0 & 0 & 1 & 814 \\
	\end{pmatrix}
	\]
	
	Добавим к 3-ой строке 4-ую, домноженную на -3:
	
	\[
	\begin{pmatrix}
	1 & -15 & -15 & 3 \\
	0 & 1 & 117 & -153 \\
	0 & 0 & 0 & -2555 \\
	0 & 0 & 1 & 814 \\
	\end{pmatrix}
	\]
	
	Поменяем местами 3-ую и 4-ую строку:
	
	\[
	\begin{pmatrix}
	1 & -15 & -15 & 3 \\
	0 & 1 & 117 & -153 \\
	0 & 0 & 1 & 814 \\
	0 & 0 & 0 & -2555 \\
	\end{pmatrix}
	\]
	
	Как видим, размерность суммы равна четырем, а значит равна размерности пространств, которую мы хотим разложить в прямую сумму \\
	Мы знаем, что сумма размерностей суммы и пересечения подпространств равна сумме размерностей исходных подпространств $\Rightarrow$ dim $V$ = 0.  \\
	Из вышесказанного получаем, что $\mathbb{R}^4 = U \oplus W$. \\
	
	
	б) Мы уже показали в предыдущем задании, что вектора, которыми заданы подпространства образуют базис в $\mathbb{R}^4$. Тогда нам достаточно просто разложить данный вектор по базису, составленному из векторов, задающих подпространства. \\
	
	Тогда нам нужно решить линейное уравнение, заданное с помощью данных векторов(записанных в столбцы), а значения координат разлагаемого вектора будут столбцом свободных членов и полученные коэффициенты будут координатами вектора в базисе:\\
		
	
	\[
	\begin{pmatrix}
	-5 & -6 & 9 & -12 & | & 4 \\
	5 & 20 & -10 & 2 & | & -5 \\
	-8 & 7 & -13 & -4 & | & -21 \\
	8 & 5 & 0 & 1 & | & 8 \\
	\end{pmatrix}
	\]
	
	Добавим к 2-ой строке 1-ую, домноженную на 1:
	
	\[
	\begin{pmatrix}
	-5 & -6 & 9 & -12 & | & 4 \\
	0 & 14 & -1 & -10 & | & -1 \\
	-8 & 7 & -13 & -4 & | & -21 \\
	8 & 5 & 0 & 1 & | & 8 \\
	\end{pmatrix}
	\]
	
	Добавим к 4-ой строке 3-ую, домноженную на 1:
	
	\[
	\begin{pmatrix}
	-5 & -6 & 9 & -12 & | & 4 \\
	0 & 14 & -1 & -10 & | & -1 \\
	-8 & 7 & -13 & -4 & | & -21 \\
	0 & 12 & -13 & -3 & | & -13 \\
	\end{pmatrix}
	\]
	
	Добавим к 3-ой строке 1-ую, домноженную на -1:
	
	\[
	\begin{pmatrix}
	-5 & -6 & 9 & -12 & | & 4 \\
	0 & 14 & -1 & -10 & | & -1 \\
	-3 & 13 & -22 & 8 & | & -25 \\
	0 & 12 & -13 & -3 & | & -13 \\
	\end{pmatrix}
	\]
	
	Добавим к 1-ой строке 3-ую, домноженную на -1:
	
	\[
	\begin{pmatrix}
	-2 & -19 & 31 & -20 & | & 29 \\
	0 & 14 & -1 & -10 & | & -1 \\
	-3 & 13 & -22 & 8 & | & -25 \\
	0 & 12 & -13 & -3 & | & -13 \\
	\end{pmatrix}
	\]
	
	Добавим к 3-ой строке 1-ую, домноженную на -1:
	
	\[
	\begin{pmatrix}
	-2 & -19 & 31 & -20 & | & 29 \\
	0 & 14 & -1 & -10 & | & -1 \\
	-1 & 32 & -53 & 28 & | & -54 \\
	0 & 12 & -13 & -3 & | & -13 \\
	\end{pmatrix}
	\]
	
	Добавим к 1-ой строке 3-ую, домноженную на -1:
	
	\[
	\begin{pmatrix}
	-1 & -51 & 84 & -48 & | & 83 \\
	0 & 14 & -1 & -10 & | & -1 \\
	-1 & 32 & -53 & 28 & | & -54 \\
	0 & 12 & -13 & -3 & | & -13 \\
	\end{pmatrix}
	\]
	
	Добавим к 3-ой строке 1-ую, домноженную на -1:
	
	\[
	\begin{pmatrix}
	-1 & -51 & 84 & -48 & | & 83 \\
	0 & 14 & -1 & -10 & | & -1 \\
	0 & 83 & -137 & 76 & | & -137 \\
	0 & 12 & -13 & -3 & | & -13 \\
	\end{pmatrix}
	\]
	
	Добавим к 3-ой строке 2-ую, домноженную на -6:
	
	\[
	\begin{pmatrix}
	-1 & -51 & 84 & -48 & | & 83 \\
	0 & 14 & -1 & -10 & | & -1 \\
	0 & -1 & -131 & 136 & | & -131 \\
	0 & 12 & -13 & -3 & | & -13 \\
	\end{pmatrix}
	\]
	
	Добавим к 2-ой строке 3-ую, домноженную на 13:
	
	\[
	\begin{pmatrix}
	-1 & -51 & 84 & -48 & | & 83 \\
	0 & 1 & -1704 & 1758 & | & -1704 \\
	0 & -1 & -131 & 136 & | & -131 \\
	0 & 12 & -13 & -3 & | & -13 \\
	\end{pmatrix}
	\]
	
	Добавим к 3-ой строке 2-ую, домноженную на 1:
	
	\[
	\begin{pmatrix}
	-1 & -51 & 84 & -48 & | & 83 \\
	0 & 1 & -1704 & 1758 & | & -1704 \\
	0 & 0 & -1835 & 1894 & | & -1835 \\
	0 & 12 & -13 & -3 & | & -13 \\
	\end{pmatrix}
	\]
	
	Добавим к 4-ой строке 2-ую, домноженную на -12:
	
	\[
	\begin{pmatrix}
	-1 & -51 & 84 & -48 & | & 83 \\
	0 & 1 & -1704 & 1758 & | & -1704 \\
	0 & 0 & -1835 & 1894 & | & -1835 \\
	0 & 0 & 20435 & -21099 & | & 20435 \\
	\end{pmatrix}
	\]
	
	Добавим к 4-ой строке 3-ую, домноженную на 11:
	
	\[
	\begin{pmatrix}
	-1 & -51 & 84 & -48 & | & 83 \\
	0 & 1 & -1704 & 1758 & | & -1704 \\
	0 & 0 & -1835 & 1894 & | & -1835 \\
	0 & 0 & 250 & -265 & | & 250 \\
	\end{pmatrix}
	\]
	
	Добавим к 3-ой строке 4-ую, домноженную на 7:
	
	\[
	\begin{pmatrix}
	-1 & -51 & 84 & -48 & | & 83 \\
	0 & 1 & -1704 & 1758 & | & -1704 \\
	0 & 0 & -85 & 39 & | & -85 \\
	0 & 0 & 250 & -265 & | & 250 \\
	\end{pmatrix}
	\]
	
	Добавим к 4-ой строке 3-ую, домноженную на 2:
	
	\[
	\begin{pmatrix}
	-1 & -51 & 84 & -48 & | & 83 \\
	0 & 1 & -1704 & 1758 & | & -1704 \\
	0 & 0 & -85 & 39 & | & -85 \\
	0 & 0 & 80 & -187 & | & 80 \\
	\end{pmatrix}
	\]
	
	Добавим к 3-ой строке 4-ую, домноженную на 1:
	
	\[
	\begin{pmatrix}
	-1 & -51 & 84 & -48 & | & 83 \\
	0 & 1 & -1704 & 1758 & | & -1704 \\
	0 & 0 & -5 & -148 & | & -5 \\
	0 & 0 & 80 & -187 & | & 80 \\
	\end{pmatrix}
	\]
	
	Добавим к 4-ой строке 3-ую, домноженную на 16:
	
	\[
	\begin{pmatrix}
	-1 & -51 & 84 & -48 & | & 83 \\
	0 & 1 & -1704 & 1758 & | & -1704 \\
	0 & 0 & -5 & -148 & | & -5 \\
	0 & 0 & 0 & -2555 & | & 0 \\
	\end{pmatrix}
	\]
	
	Разделим 4-ую строку на -2555:
	
	\[
	\begin{pmatrix}
	-1 & -51 & 84 & -48 & | & 83 \\
	0 & 1 & -1704 & 1758 & | & -1704 \\
	0 & 0 & -5 & -148 & | & -5 \\
	0 & 0 & 0 & 1 & | & 0 \\
	\end{pmatrix}
	\]
	
	Добавим к 3-ой строке 4-ую, домноженную на 148:
	
	\[
	\begin{pmatrix}
	-1 & -51 & 84 & -48 & | & 83 \\
	0 & 1 & -1704 & 1758 & | & -1704 \\
	0 & 0 & -5 & 0 & | & -5 \\
	0 & 0 & 0 & 1 & | & 0 \\
	\end{pmatrix}
	\]
	
	Добавим к 2-ой строке 4-ую, домноженную на -1758:
	
	\[
	\begin{pmatrix}
	-1 & -51 & 84 & -48 & | & 83 \\
	0 & 1 & -1704 & 0 & | & -1704 \\
	0 & 0 & -5 & 0 & | & -5 \\
	0 & 0 & 0 & 1 & | & 0 \\
	\end{pmatrix}
	\]
	
	Добавим к 1-ой строке 4-ую, домноженную на 48:
	
	\[
	\begin{pmatrix}
	-1 & -51 & 84 & 0 & | & 83 \\
	0 & 1 & -1704 & 0 & | & -1704 \\
	0 & 0 & -5 & 0 & | & -5 \\
	0 & 0 & 0 & 1 & | & 0 \\
	\end{pmatrix}
	\]
	
	Разделим 3-ую строку на -5:
	
	\[
	\begin{pmatrix}
	-1 & -51 & 84 & 0 & | & 83 \\
	0 & 1 & -1704 & 0 & | & -1704 \\
	0 & 0 & 1 & 0 & | & 1 \\
	0 & 0 & 0 & 1 & | & 0 \\
	\end{pmatrix}
	\]
	
	Добавим к 2-ой строке 3-ую, домноженную на 1704:
	
	\[
	\begin{pmatrix}
	-1 & -51 & 84 & 0 & | & 83 \\
	0 & 1 & 0 & 0 & | & 0 \\
	0 & 0 & 1 & 0 & | & 1 \\
	0 & 0 & 0 & 1 & | & 0 \\
	\end{pmatrix}
	\]
	
	Добавим к 1-ой строке 3-ую, домноженную на -84:
	
	\[
	\begin{pmatrix}
	-1 & -51 & 0 & 0 & | & -1 \\
	0 & 1 & 0 & 0 & | & 0 \\
	0 & 0 & 1 & 0 & | & 1 \\
	0 & 0 & 0 & 1 & | & 0 \\
	\end{pmatrix}
	\]
	
	Добавим к 1-ой строке 2-ую, домноженную на 51:
	
	\[
	\begin{pmatrix}
	-1 & 0 & 0 & 0 & | & -1 \\
	0 & 1 & 0 & 0 & | & 0 \\
	0 & 0 & 1 & 0 & | & 1 \\
	0 & 0 & 0 & 1 & | & 0 \\
	\end{pmatrix}
	\]
	
	Домножим 1-ую строку на -1:
	
	\[
	\begin{pmatrix}
	1 & 0 & 0 & 0 & | & 1 \\
	0 & 1 & 0 & 0 & | & 0 \\
	0 & 0 & 1 & 0 & | & 1 \\
	0 & 0 & 0 & 1 & | & 0 \\
	\end{pmatrix}
	\]
	
	Откуда получаем, что: \\$x = u_1 + u_3$.
	
	
$Q(x_1,x_2,x_3) = 4 \cdot x_1^2 - 3 \cdot x_3^2 - 4 \cdot x_1x_2 + 12 \cdot x_1x_3 + 2 \cdot x_2x_3 = 
4\bigl[x_1^2 +2x_1(3x_3 - x_2) + (3x_3 - x_2)^2 \bigr] - 4(3x_3 - x_2)^2 - 3 \cdot x_3^2 + 2 \cdot x_2x_3$. \\
Пусть $y_1 = x_1 + 3x_3 - x_2$, $y_2 = x_2$, $y_3 = x_3$ \\

$x = S_1 \cdot y$.
\[ S_1 =
\begin{pmatrix}
1 & 1 & -3 \\
0 & 1 & 0 \\
0 & 0 & 1 \\
\end{pmatrix}
\]

$Q = 4y_1^2 -36y_3^2 + 24y_2y_3 - 4y_2^2 - 3 \cdot y_3^2 + 2 \cdot y_2y_3 = 4y_1^2 -39y_3^2 + 26y_2y_3 - 4y_2^2 = 4y_1^2 - 4\bigl[ y_2^2 - 6.5y_2y_3 + 3.25y_3^2 \bigr] + 3.25y_3^2$.\\
Пусть $z_1 = y_1$, $z_2 = y_2 - 3.25y_3$, $z_3 = y_3$. \\

$y = S_2 \cdot z$. 

\[ S_2 =
\begin{pmatrix}
1 & 0 & 0 \\
0 & 1 & 3.25 \\
0 & 0 & 1 \\
\end{pmatrix}
\]

$Q = 4z_1^2 - z_2^2 + 4z_3^2$.

\[ A =
\begin{pmatrix}
4 & -2 & 6 \\
-2 & 0 & 1 \\
6 & 1 & -3 \\
\end{pmatrix}
\]

\[ S =
\begin{pmatrix}
1 & 1 & 1 \\
0 & 1 & 4 \\
0 & 0 & 1 \\
\end{pmatrix}
\]

\[ D =
\begin{pmatrix}
4 & -2 & 6 \\
2 & -2 & 7 \\
2 & -1 & 13 \\
\end{pmatrix}
\]

	\end{document}