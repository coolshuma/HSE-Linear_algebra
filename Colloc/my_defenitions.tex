\documentclass[a4paper,12pt]{article}
\usepackage{amsmath}
\usepackage{cmap}					% поиск в PDF
\usepackage{mathtext} 				% русские буквы в формулах
\usepackage[T2A]{fontenc}			% кодировка
\usepackage[utf8]{inputenc}			% кодировка исходного текста
\usepackage[english,russian]{babel}	% локализация и переносы

% Изменим формат \section и \subsection:
\usepackage{titlesec}
\titleformat{\section}
{\vspace{1cm}\centering\LARGE\bfseries}	% Стиль заголовка
{}										% префикс
{0pt}									% Расстояние между префиксом и заголовком
{} 										% Как отображается префикс
\titleformat{\subsection}				% Аналогично для \subsection
{\Large\bfseries}
{}
{0pt}
{}

%% Отступы между абзацами и в начале абзаца 
\setlength{\parindent}{0pt}
\setlength{\parskip}{\medskipamount}

% Перенос знаков в формулах (по Львовскому)
\newcommand*{\hm}[1]{#1\nobreak\discretionary{}
{\hbox{$\mathsurround=0pt #1$}}{}}

%% Изменяем размер полей
\usepackage[top=0.5in, bottom=0.75in, left=0.625in, right=0.625in]{geometry}

\title{Линейная Алгебра и Геометрия\\Определения}
\begin{document}
\maketitle

\begin{enumerate}
\item \textbf{Совместные и несовместные СЛУ.} 

Совместная СЛУ -- это СЛУ, которая имеет хотя бы одно решение. \\
Несовместная СЛУ -- не имеет решений. 

\item \textbf{Эквивалентные СЛУ.} 

Это СЛУ, которые имеют одинакавое множества решений.

\item \textbf{Расширенная матрица системы линейных уравнений.} 
 
Матрица вида $(A | \vec b)$. Где $A$ -- матрица коэффициентов, $\vec b$ -- вектор свободных членов.

\item \textbf{Элементарные преобразования матриц.}  --

Это такие преобразования в результате которых не меняется эквивалентность матриц, т.е. множество решений СЛУ, которому соответствует данная матрица. 
\begin{enumerate}
	\item Прибавление к одной строке матрицы другой строки, домноженной на некоторый коэффициент.
	\item Перестановка двух строк матрицы.
	\item Умножение строки матрицы на некоторый коэффициент.
\end{enumerate}

\item \textbf{Ступенчатый вид матрицы.}
\begin{enumerate}
	\item Номера ведущих элементов(первый ненулевой) строк строго возрастают.
	\item Все нулевые строки стоят в конце.
\end{enumerate}

\item \textbf{Улучшенный ступенчатый вид матрицы.}

\begin{enumerate}
	\item Имеет ступенчатый вид.
	\item Все ведущие элементы строк равны 1 и во всех столбцах, содержащих ведущие элементы, все остальные элементы равны нулю
\end{enumerate}

\item \textbf{Теорема о каноническом виде, к которому можно привести матрицу при помощи элементарных преобразований строк.}
\begin{Theorem}
	Всякую матрицу элементарными преобразованиями можно привести к каноническому виду.
\end{Theorem}

\item \textbf{Общее решение совместной системы уравнений.}

\textbf{Преамбула}:
Когда в матрице, соответствующей СЛУ ненулевых строк меньше, чем неизвестных. \\
Тогда назовем главными те неизвестные, коэффициенты при которых являются лидерами строк, а остальные назовем свободными. Отбросив нулевые строки и перенеся члены со свободными неизвестными в правую часть, мы снова получим строго треугольную систему. Решая ее как в предыдущем случае, находим выражение главных неизвестных через свободные. Подставляя в свободные любые значения, получаем бесконечное количество решений — система будет неопределенной.\\
\textbf{Суть:}\\
\textbf{Общее решение совместной СЛУ} -- это множество всех решений этой системы. 

\item \textbf{Сколько может быть решений у СЛУ с действительными коэффициентами.}
\begin{enumerate}
	\item Если система \textit{несовместна}, то она не имеет решений.
	\item Если система \textit{совместна и определена}, то она имеет одно решение.
	\item Если система \textit{совместна и неопределена}, то она имеет бесконечно много решений.
\end{enumerate}

\item \textbf{Однородная СЛУ. Что можно сказать про ее множество решений?} --

это СЛУ вида $A\vec{x} = 0$, т.е. СЛУ у которой все правые части уравнений равны нулю. \\
О ней можно сказать то, что она всегда совместна, т.к. всегда имеет как минимум одно решение: $\vec{x} = 0$.

\begin{Theorem}
	Пусть $c$ — какое-то решение неоднородной системы, а $L$ -- множество всех решений связанной с ней однородной системы. Тогда $c+L$ есть множество всех решений неоднородной системы.
\end{Theorem}

\item \textbf{Свойство однородной СЛУ, у которой число неизвестных больше, чем число уравнений.} 

	У такой СЛУ при приведении к ступенчатому виду будет хотя бы одна свободная неизвестная $x_i$. Значит СЛУ имеет бесконечно много решений, среди которых есть ненулевые. 
	
\item \textbf{Сумма двух матриц и умножение матрицы на скаляр.} 
\begin{Comment}
	Говоря <<матрица $A$ размера $m \times n$>>  мы подразумеваем, что в матрице $m$ строк и $n$ столбцов. Это можно обозначить следующим образом: $A \in Mat_{m \times n}$.
\end{Comment}

	\textit{\textbf{Суммой}} двух матриц $A$ и $B$ размера $m \times n$ называется такая матрица $C$ размера $m \times n$, в которой каждый элемент равен сумме соответствующих элементов матриц $A$ и $B$, т.е.:
	\[
		c_{ij} = a_{ij} + b_{ij}
	\]
	\begin{Comment}
		Сложение двух матриц определено только в том случае, когда их размеры одинаковы
	\end{Comment}
	Свойства сложения матриц:
	\begin{enumerate}
		\item $\forall A, B, C \in Mat_{m\times n}\ (A + B) + C = A + (B + C)$ -- ассоциативность.
		\item $\forall A, B \in Mat_{m\times n}\ A + B= B + A$ -- коммутативность.
		\item $\exists\ 0 \in Mat_{m\times n}\ : \forall A \in Mat_{m\times n}\ A + 0= 0 + A = A$ -- сложение с нулевой матрицей(существование нейтрального элемента по сложению)
		\item $\forall A \in Mat_{m\times n}\ \exists !\ (-A) = (-a_{ij})\ :  A + (-A)= (-A) + A = 0$ -- существование противоположной матрицы.
	\end{enumerate}
	$\exists!$ -- существует и притом только один.\\\\
	
	\textit{\textbf{Произведением}} матрицы $A$ размера $m \times n$ на \textit{\textbf{скаляр}} $\lambda$ называется такая матрица $B$ размера $m \times n$, в которой каждый элемент равен произведению соответствующего элемента матрицы $A$ и скаляра $\lambda$, т.е.:
	\[
	b_{ij} = \lambda \cdot a_{ij}
	\]
	Свойства умножения матриц на скаляр:
	\begin{enumerate}
		\item $\forall A \in Mat_{m\times n}\ 1 \cdot A = A$ -- умножение на единичный скаляр.(существование нейтрального элемента по умножению)
		\item $\forall\ \alpha, \beta \in \mathbb{R}, A \in Mat_{m\times n}\ (\alpha\beta)A= \alpha(\beta A)$ -- ассоциативность.
		\item $\forall\ \alpha, \beta \in \mathbb{R}, A \in Mat_{m\times n}\ (\alpha+\beta)A= \alpha A + \beta A$ -- дистрибутивность относительно скаляров.
		\item $\forall \l \in \R, \forall A, B \in Mat_{m\times n}\ \l(A+B) = \l A + \l B$ --дистрибутивность относительно матриц.
	\end{enumerate}
	
	\item \textbf{Транспонированная матрица.} --
	
	Это матрица, над которой проведено преобразование, при котором стобцы становятся строками и наоборот, т.е.: 
	\[
		(a_{ij})^T = a_{ji}
	\]
	
	Обозначается $A^T$.\\
	Свойства транспонирования:
	\begin{enumerate}
		\item $(\l A)^T = \l(A)^T$ -- связь с умножением на скаляр.
		\item $(A + B)^T = A^T + B^T$ -- связь со сложением матриц.
		\item $(AB)^T = B^T A^T$
	\end{enumerate}
	
	\item \textbf{Произведение двух матриц.}
	
	\textit{\textbf{Произведением}} двух матриц $A \in Mat_{m \times l}$ и $B \in Mat_{l \times n}$ называется такая матрица $C \in Mat_{m \times n}$ размера $m \times n$, в которой каждый элемент $(с_{ij})$ равен сумме произведений элементов i-ой строки матрицы $A$ на элементы j-ый столбец матрицы $B$, т.е.:
	\[
		c_{ij} = \sum_{k=1}^{l} a_{ik}b_{kj} = a_{i1}b_{1j} + a_{i2}b_{2j} + \ldots + a_{il}b{nl}.
	\]
	Свойства произведения матриц:
	\begin{enumerate}
		\item Некоммутативно в общем случае.
		\item $A(B+C) = AB + AC$ -- левая дистрибутивность.
		\item $(A+B)C = AC+BC$ -- правая дистрибутивность.
		\item $\l (A+B) = (\l A)B = A (\l B)$.
		\item $A(BC) = A(BC)$ -- ассоциативность.
	\end{enumerate}
	
	\item \textbf{Диагональная матрица.} --
	
	Квадратная матрица, элементы которой вне главной диагонали равны нулю. Обозначается как $diag(a_{11}, a_{22}, a{33}, a{44} \ldots a_{nn})$.
	
	\item \textbf{Единичная матрица, ее свойства.} --
	
	квадратная матрица, у которой элементы главной диагонали равны единицы, а остальные нулю, т.е. $E_n = diag(1,1,1 \ldots 1)$. 
	
	\textit{\textbf{Основное свойство:}}\\
	$\forall A \in Mat_{n \times n}\ AE = EA = A$.
	
	Еще: 	
	\begin{enumerate}
		\item $\forall A \in Mat_{n \times n}\ A^0 = E$.
		\item $\forall A \in Mat_{n \times n}\ A A^{1} = E$.
		\item $det\ E = 1$
	\end{enumerate}
	
	\item \textbf{След квадратной матрицы $A$.} --
	
	Это сумма всех стоящих на главной диагонали матрицы $A$ элементов. \\
	Обозначается $tr\ A$ (от английского слова <<trace>> -- след).
	Свойства следа:
	\begin{enumerate}
		\item $tr(\l A) = \l tr(A)$.
		\item $tr(A + B)^T = trA + trB$
		\item $tr(A)^T = trA$
	\end{enumerate}
	
	\item \textbf{След произведения двух матриц.}
	
	$\forall A \in Mat_{m \times n}, B \in Mat_{n \times m} trAB = tr BA$.
	\begin{proof}
		Пусть $AB = X, BA = Y$, тогда:
		\[
		tr X = \sum_{k=1}^{m} x_{kk} = \sum_{k=1}^{m}\left(\sum_{l=1}^{n} a_{kl}b_{lk}\right) = \sum_{l=1}^{n} \left( \sum_{k=1}^{m}b_{lk}a_{kl} \right) = \sum_{l=1}^{n}y_{ll} = tr Y.
		\]
	\end{proof}
	
	\item \textbf{Перестановки и подстановки элементов множества \{1, 2, $\ldots$ n\}.}
	
	\textit{\textbf{Перестановкой}} из $n$ элементов множества \{1, 2, $\ldots$ n\} называется всякий упорядоченный набор, в котором каждый элемент присутствует ровно один раз.\\
	\textit{\textbf{Подстановкой}} -- биективное отображение из множества \{1, 2, $\ldots$ n\} в него же. Обозначение:
	\[ \sigma = 
	\begin{pmatrix}
	x_1 & x_2 & \ldots & x_n \\
	\sigma(x_1) & \sigma(x_2) & \ldots & \sigma(x_n) \\
	\end{pmatrix}
	\]
	Говорят, что $x_1$ переходит в $sigma(x_1)$, $x_2$ переходит в $sigma(x_2)$ и т.д.
	
	\item \textbf{Инверсия в подстановке. Знак подстановки. Чётные и нечётные подстановки.}
	
	\textit{\textbf{Инверсия}} -- это такая пара индексов $i$ и $j$, что $i<j$, но $\sigma(i) > \sigma(j)$.\\
	Пусть число инверсий в подстановке $\sigma = N(\sigma)$. Тогда \textit{\textbf{знак подстановки}} $(-1)^{N(\sigma)}$. \\
	Подстановка \textit{\textbf{чётна}}, если ее знак равен 1 и нечетна иначе.
	
	\item \textbf{Произведение двух подстановок.} -- 
	это новая подстановка степени $n$, получившаяся в результате последовательного(справа налево!) применения двух перестановок степени $n$.
	Свойства произведения подстановок: 
	\begin{enumerate}
		\item $(\sigma_1 \sigma_2) \sigma_3 = \sigma_1 (\sigma_2 \sigma_3)$ -- ассоциативность.
		\item Подстановки степени больше двух некоммутативны.
		\item $det\ E = 1$
	\end{enumerate}
		
	\item \textbf{Тождественная подстановка, обратная подстановка.}
	
	\textit{\textbf{Тождественная подстановка}} -- переводит элементы сами в себя, обозначается id. Т.е. $id(x) = x, \forall x$.
	
	Свойство:\\
	$\forall\ \sigma\ \  id \cdot \sigma = \sigma \cdot id = id$. 
	
	\textit{\textbf{Обратная подстановка}} -- такая подстановка $\sigma^{-1}$, что $\sigma \cdot \sigma^{-1} = \sigma^{-1} \cdot \sigma = id.$ \\
	Свойство:\\
	$sgn(\sigma) = sgn(\sigma^{-1}).$ -- Знак обратной подстановки равен знаку исходной.
	
	\item \textbf{Транспозиция, элементарная транспозиция.}
	
	\textit{\textbf{Транспозиция}} -- это такая подстановка, которая меняет ровно два элемента местами.
	\textit{\textbf{Элементарня транспозиция}} -- это такая транспозиция, которая меняет местами два соседних элемента.
	
	
	\item \textbf{Поведение знака подстановки при умножении справа на транспозицию. ??Знак транспозиции??.}
	
	Пусть $\tau$ -- транспозиция, тогда $sgn(\sigma \tau) = -sgn(\sigma)$.
	
	\item \textbf{Теорема о знаке произведения двух подстановок.}
	\begin{Theorem}
		Знак произведения подстановок есть произведение знаков подстановок. $\sgn(\sigma \rho) = sgn(\sigma) \cdot sgn(\rho)$.
	\end{Theorem}

\end{enumerate}

\end{document}
