\documentclass[a4paper,12pt]{article}
\usepackage{amsmath}
\usepackage{cmap}					% поиск в PDF
\usepackage{mathtext} 				% русские буквы в формулах
\usepackage[T2A]{fontenc}			% кодировка
\usepackage[utf8]{inputenc}			% кодировка исходного текста
\usepackage[english,russian]{babel}	% локализация и переносы

% Изменим формат \section и \subsection:
\usepackage{titlesec}
\titleformat{\section}
{\vspace{1cm}\centering\LARGE\bfseries}	% Стиль заголовка
{}										% префикс
{0pt}									% Расстояние между префиксом и заголовком
{} 										% Как отображается префикс
\titleformat{\subsection}				% Аналогично для \subsection
{\Large\bfseries}
{}
{0pt}
{}

%% Отступы между абзацами и в начале абзаца 
\setlength{\parindent}{0pt}
\setlength{\parskip}{\medskipamount}

% Перенос знаков в формулах (по Львовскому)
\newcommand*{\hm}[1]{#1\nobreak\discretionary{}
{\hbox{$\mathsurround=0pt #1$}}{}}

%% Изменяем размер полей
\usepackage[top=0.5in, bottom=0.75in, left=0.625in, right=0.625in]{geometry}
\begin{document}
	\section{Лист 1} 
	\subsection{Задача 1} 
	\textit{(1)}\\
	Доказать, что $[A, B] = -[B, A]\  \forall A,B \in L;$\\
	$[A, B] = AB - BA.$\\
	$[B, A] = BA - AB.$\\
	$-[B, A] = AB - BA = [A, B].$\\
	\textit{(2)}\\
	Доказать, что $[A, [B, C]] = [[A, B], C] + [B, [A, C]]$.\\\\
	$[A, [B, C]] = [A, BC - CB] = ABC - ACB - BCA + CBA.$\\\\
	$[[A, B], C] + [B, [A, C]] = [AB - BA, C] + [B, AC - CA] = (ABC - BAC - CAB + CBA) + (BAC - BCA - ACB + CAB) = ABC + CBA - BCA - ACB = [A, [B, C]].$\\
	
	\subsection{Задача 2} 
	Множество кососиметрических матриц непусто и входит в $M_n$. Тогда докажем, что множество кососиметрических матриц удовлетворяет трем условиям: \\
	(1) $A+B \in L \ \forall A,B \in L$.\\
	(2) $\lambda A \in L \ \forall A \in L\  и\  \lambda \in \mathbb{R}$.\\
	(3) $[A,B] \in L \ \forall A,B \in L$.\\
	Доказательства:\\
	\textit{(1)}\\
	Пусть $A+B = C$.\\
	В кососиметрических матрицах $a_{ij} = -a_{ji}$, тогда и $b_{ij} = -b_{ji}$.\\
	Тогда:\\
	 $c_{ij} = a_{ij} + b_{ij} = -a_{ji} - b_{ji}.$\\
	$c_{ji} = a_{ji} + b_{ji} = -c_{ij}.$\\
	\textit{(2)}\\
	 $a_{ij} = -a_{ji}.\\ \lambda \cdot a_{ij} = \lambda \cdot -a_{ji}$\\
	Т.е. $\lambda \cdot a_{ij} = -\lambda \cdot a_{ji}$, а значит матрица домноженая на $\lambda$ тоже кососиметричная и тоже принадлежит L. \\
	\textit{(3)}\\
	$[A, B] = AB - BA.$
	$(AB)^T = B^T \cdot A^T$
	По определению $A^T = -A, B^T = -B$.\\
	Тогда $BA = (-B)^T \cdot (-A)^T = ((-A) \cdot (-B))^T = (AB)^T$\\
	Значит $[A, B] = AB - (AB)^T = E$.\\
	Пусть $AB = C$ и $(AB)^T = D$.\\
	Тогда $c_{ij} = d_{ji}$.\\
	$e_{ij} = c_{ij} - d_{ij} = c_{ij} - c_{ji}$.\\
	$e_{ji} = c_{ji} - d_{ji} = c_{ji} - c_{ij} = - e_{ji}$.
	
	%Пусть AB = C и BA = D.\\
	%\[
	%	c_{ij} = \sum_{t = 1}^{n} a_{it} \cdot b_{tj}
	%\]
	%\[
	%	c_{ji} = \sum_{t = 1}^{n} a_{jt} \cdot b_{ti}
	%\]
	%\[
	%	d_{ij} = \sum_{t = 1}^{n} b_{it} \cdot a_{tj}
	%\]
	%\[
	%	d_{ji} = \sum_{t = 1}^{n} b_{jt} \cdot a_{ti}
	%\]
	
	\subsection{Задача 5}
	Докажем три свойства: \\
	(1) $A+B \in I \ \forall A,B \in I$.\\
	(2) $\lambda A \in I \ \forall A \in I\  и\  \lambda \in \mathbb{R}$.\\
	(3) $[A,B] \in I \ \forall A \in I, B \in L$.\\
	\textit{(1)}\\
	Если $A, B$ принадлежат центру L, то и A + B также принaдлежат центру, а значит принадлежат и идеалу, поскольку $\forall X \in L \Rightarrow AX = 0, BX = 0 \Rightarrow (A+B)X = 0$.\\
	\textit{(2)}\\
	Если $A$ принадлежат центру L, то и $\lambda A$ также принaдлежит центру, а значит принадлежит и идеалу, поскольку $\forall X \in L \Rightarrow AX = 0 \Rightarrow \lambda AX = 0$. \\
	\textit{(3)}\\
	Т.к. центр L составляют матрицы, произведение которых с любой другой матрицей из L равно нулю, то $AB = 0$, а значит принадлежит  идеалу, т.к. $0 \in I$.\\\\\\\\\\\\\\\\\\
	
	\subsection{Задача 5}
	Докажем три свойства: \\
	(1) $A+B \in I \ \forall A,B \in I$.\\
	(2) $\lambda A \in I \ \forall A \in I\  и\  \lambda \in \mathbb{R}$.\\
	(3) $[A,B] \in I \ \forall A \in L, B \in I$.\\
	\textit{(1)}\\
	Согласно определению коммутанта A представима в виде суммы конечного числа коммутаторов матриц из L, назовем эту сумму $S_1$. Точно так же B представима в виде суммы конечного числа коммутаторов матриц из L, назовем эту сумму $S_2$. Тогда A + B = $S_1 + S_2$ и, так как сумма двух конечнох чисел также конечна и все коммутаторы в $S1\  и\  S2$ -- это коммутаторы матриц из L, то и A + B также будет являть собой сумму конечного числа коммутаторов матриц из L, а значит входить в комутант данного L $\Rightarrow$ входить и в идеал. \\
	\textit{(2)}\\
	Сумма, умноженная на некоторое $\lambda \in \mathbb{R}$ конечна, а также может быть представленна в виде суммы членов, составляющих первоначальную сумму. Достаточно просто взять каждый член $\lambda$ раз. Значит новая сумма представима в виде конечного числа коммутаторов матриц из L, а значит входит в комутант данного L $\Rightarrow$ входит и в идеал. \\
	\textit{(3)}\\
	$[S_1, S_2] = S_1 \cdot S_2 - S_2 \cdot S_1$.
	
	\end{document}