\documentclass[a4paper,12pt]{article}
\usepackage{amsmath}
\usepackage{cmap}					% поиск в PDF
\usepackage{mathtext} 				% русские буквы в формулах
\usepackage[T2A]{fontenc}			% кодировка
\usepackage[utf8]{inputenc}			% кодировка исходного текста
\usepackage[english,russian]{babel}	% локализация и переносы

% Изменим формат \section и \subsection:
\usepackage{titlesec}
\titleformat{\section}
{\vspace{1cm}\centering\LARGE\bfseries}	% Стиль заголовка
{}										% префикс
{0pt}									% Расстояние между префиксом и заголовком
{} 										% Как отображается префикс
\titleformat{\subsection}				% Аналогично для \subsection
{\Large\bfseries}
{}
{0pt}
{}

%% Отступы между абзацами и в начале абзаца 
\setlength{\parindent}{0pt}
\setlength{\parskip}{\medskipamount}

% Перенос знаков в формулах (по Львовскому)
\newcommand*{\hm}[1]{#1\nobreak\discretionary{}
{\hbox{$\mathsurround=0pt #1$}}{}}

%% Изменяем размер полей
\usepackage[top=0.5in, bottom=0.75in, left=0.625in, right=0.625in]{geometry}
\begin{document}
	\section{Домашнее задание к 15.12 \\ Шумилкин Андрей. Группа 163} 
	\subsection{Задача 665}
	\[
	\begin{pmatrix}
	2 & 3 & 1 & 7 \\
	3 & 7 & -6 & -2 \\
	5 & 8 & 1 & 0 \\
	\end{pmatrix}
	\]
	
	Добавим к 2-ой строке 1-ую, домноженную на -1:
	\[
	\begin{pmatrix}
	2 & 3 & 1 & 7 \\
	1 & 4 & -7 & -9 \\
	5 & 8 & 1 & 0 \\
	\end{pmatrix}
	\]
	
	Добавим к 1-ой строке 2-ую, домноженную на -1:
	\[
	\begin{pmatrix}
	1 & -1 & 8 & 16 \\
	1 & 4 & -7 & -9 \\
	5 & 8 & 1 & 0 \\
	\end{pmatrix}
	\]
	
	Добавим к 2-ой строке 1-ую, домноженную на -1:
	\[
	\begin{pmatrix}
	1 & -1 & 8 & 16 \\
	0 & 5 & -15 & -25 \\
	5 & 8 & 1 & 0 \\
	\end{pmatrix}
	\]
	
	Добавим к 3-ой строке 1-ую, домноженную на -5:
	\[
	\begin{pmatrix}
	1 & -1 & 8 & 16 \\
	0 & 5 & -15 & -25 \\
	0 & 13 & -39 & -80 \\
	\end{pmatrix}
	\]
	
	Разделим 2-ую строку на 5:
	\[
	\begin{pmatrix}
	1 & -1 & 8 & 16 \\
	0 & 1 & -3 & -5 \\
	0 & 13 & -39 & -80 \\
	\end{pmatrix}
	\]
	
	Добавим к 3-ой строке 2-ую, домноженную на -13:
	\[
	\begin{pmatrix}
	1 & -1 & 8 & 16 \\
	0 & 1 & -3 & -5 \\
	0 & 0 & 0 & -15 \\
	\end{pmatrix}
	\]
	
	
	\subsection{Задача 612}
	 \[
	 \begin{pmatrix}
	 0 & 4 & 10 & 1 \\
	 2 & 2 & 4 & 3 \\
	 1 & 7 & 17 & 3 \\
	 3 & 1 & 1 & 4 \\
	 \end{pmatrix}
	 \]
	 
	 Добавим к 4-ой строке 3-ую, домноженную на -1:
	 \[
	 \begin{pmatrix}
	 0 & 4 & 10 & 1 \\
	 2 & 2 & 4 & 3 \\
	 1 & 7 & 17 & 3 \\
	 2 & -6 & -16 & 1 \\
	 \end{pmatrix}
	 \]
	 
	 Добавим к 3-ой строке 4-ую, домноженную на -3:
	 \[
	 \begin{pmatrix}
	 0 & 4 & 10 & 1 \\
	 2 & 2 & 4 & 3 \\
	 -5 & 25 & 65 & 0 \\
	 2 & -6 & -16 & 1 \\
	 \end{pmatrix}
	 \]
	 
	 Добавим к 2-ой строке 4-ую, домноженную на -3:
	 \[
	 \begin{pmatrix}
	 0 & 4 & 10 & 1 \\
	 -4 & 20 & 52 & 0 \\
	 -5 & 25 & 65 & 0 \\
	 2 & -6 & -16 & 1 \\
	 \end{pmatrix}
	 \]
	 
	 Добавим к 1-ой строке 4-ую, домноженную на -1:
	 \[
	 \begin{pmatrix}
	 -2 & 10 & 26 & 0 \\
	 -4 & 20 & 52 & 0 \\
	 -5 & 25 & 65 & 0 \\
	 2 & -6 & -16 & 1 \\
	 \end{pmatrix}
	 \]
	 
	 Разделим 3-ую строку на 5:
	 \[
	 \begin{pmatrix}
	 -2 & 10 & 26 & 0 \\
	 -4 & 20 & 52 & 0 \\
	 -1 & 5 & 13 & 0 \\
	 2 & -6 & -16 & 1 \\
	 \end{pmatrix}
	 \]
	 
	 Разделим 2-ую строку на 4:
	 \[
	 \begin{pmatrix}
	 -2 & 10 & 26 & 0 \\
	 -1 & 5 & 13 & 0 \\
	 -1 & 5 & 13 & 0 \\
	 2 & -6 & -16 & 1 \\
	 \end{pmatrix}
	 \]
	 
	 Добавим к 2-ой строке 3-ую, домноженную на -1:
	 \[
	 \begin{pmatrix}
	 -2 & 10 & 26 & 0 \\
	 0 & 0 & 0 & 0 \\
	 -1 & 5 & 13 & 0 \\
	 2 & -6 & -16 & 1 \\
	 \end{pmatrix}
	 \]
	 
	 Поменяем местами 1-ую и 2-ую строку:
	 \[
	 \begin{pmatrix}
	 0 & 0 & 0 & 0 \\
	 -2 & 10 & 26 & 0 \\
	 -1 & 5 & 13 & 0 \\
	 2 & -6 & -16 & 1 \\
	 \end{pmatrix}
	 \]
	 
	 Добавим к 2-ой строке 3-ую, домноженную на -2:
	 \[
	 \begin{pmatrix}
	 0 & 0 & 0 & 0 \\
	 0 & 0 & 0 & 0 \\
	 -1 & 5 & 13 & 0 \\
	 2 & -6 & -16 & 1 \\
	 \end{pmatrix}
	 \]
	 
	  \subsection{Задача 668}
	 
	 \[
	 \begin{pmatrix}
	 3 & 2 & 5 & 1 \\
	 2 & 4 & 6 & 3 \\
	 5 & 7 & \lambda & 5 \\
	 \end{pmatrix}
	 \]
	 
	 Добавим к 1-ой строке 2-ую, домноженную на -1:
	 \[
	 \begin{pmatrix}
	 1 & -2 & -1 & -2 \\
	 2 & 4 & 6 & 3 \\
	 5 & 7 & \lambda & 5 \\
	 \end{pmatrix}
	 \]
	 
	 Добавим к 2-ой строке 1-ую, домноженную на -2:
	 \[
	 \begin{pmatrix}
	 1 & -2 & -1 & -2 \\
	 0 & 8 & 8 & 7 \\
	 5 & 7 & \lambda & 5 \\
	 \end{pmatrix}
	 \]
	 
	 Добавим к 3-ой строке 1-ую, домноженную на -5:
	 \[
	 \begin{pmatrix}
	 1 & -2 & -1 & -2 \\
	 0 & 8 & 8 & 7 \\
	 0 & 17 & \lambda + 5 & 15 \\
	 \end{pmatrix}
	 \]
	 
	 Добавим к 3-ой строке 2-ую, домноженную на -2:
	 \[
	 \begin{pmatrix}
	 1 & -2 & -1 & -2 \\
	 0 & 8 & 8 & 7 \\
	 0 & 1 & \lambda-11 & 1 \\
	 \end{pmatrix}
	 \]
	 
	 Поменяем местами 2-ую и 3-ую строку:
	 \[
	 \begin{pmatrix}
	 1 & -2 & -1 & -2 \\
	 0 & 1 & \lambda-11 & 1 \\
	 0 & 8 & 8 & 7 \\
	 \end{pmatrix}
	 \]
	 
	 Добавим к 3-ой строке 2-ую, домноженную на -8:
	 \[
	 \begin{pmatrix}
	 1 & -2 & -1 & -2 \\
	 0 & 1 & \lambda-11 & 1 \\
	 0 & 0 & -8\lambda+96 & -1 \\
	 \end{pmatrix}
	 \]
	 
	 Разделим 3-юю строку на 8:
	 \[
	 \begin{pmatrix}
	 1 & -2 & -1 & -2 \\
	 0 & 1 & \lambda-11 & 1 \\
	 0 & 0 & -\lambda+12 & -1/8 \\
	 \end{pmatrix}
	 \]
	 
	 Добавим к 2-ой строке 3-ую, домноженную на 1:
	 \[
	 \begin{pmatrix}
	 1 & -2 & -1 & -2 \\
	 0 & 1 & 1 & 1 \\
	 0 & 0 & -\lambda+12 & -1/8 \\
	 \end{pmatrix}
	 \]
	 
	 При $\lambda \not = -12$ вектор будет выражаем.
	 
	 
	 
	 \subsection{Задача 621}
	 \[
	 \begin{pmatrix}
	 24 & 19 & 36 & 72 & -38 \\
	 49 & 40 & 73 & 147 & -80 \\
	 73 & 59 & 98 & 219 & -118 \\
	 47 & 36 & 71 & 141 & -72 \\
	 \end{pmatrix}
	 \]
	 
	 Поменяем местами 1-ую и 2-ую строку:
	 \[
	 \begin{pmatrix}
	 49 & 40 & 73 & 147 & -80 \\
	 24 & 19 & 36 & 72 & -38 \\
	 73 & 59 & 98 & 219 & -118 \\
	 47 & 36 & 71 & 141 & -72 \\
	 \end{pmatrix}
	 \]
	 
	 Добавим к 1-ой строке 2-ую, домноженную на -2:
	 \[
	 \begin{pmatrix}
	 1 & 2 & 1 & 3 & -4 \\
	 24 & 19 & 36 & 72 & -38 \\
	 73 & 59 & 98 & 219 & -118 \\
	 47 & 36 & 71 & 141 & -72 \\
	 \end{pmatrix}
	 \]
	 
	 Добавим к 2-ой строке 1-ую, домноженную на -24:
	 \[
	 \begin{pmatrix}
	 1 & 2 & 1 & 3 & -4 \\
	 0 & -29 & 12 & 0 & 58 \\
	 73 & 59 & 98 & 219 & -118 \\
	 47 & 36 & 71 & 141 & -72 \\
	 \end{pmatrix}
	 \]
	 
	 Добавим к 3-ой строке 1-ую, домноженную на -73:
	 \[
	 \begin{pmatrix}
	 1 & 2 & 1 & 3 & -4 \\
	 0 & -29 & 12 & 0 & 58 \\
	 0 & -87 & 25 & 0 & 174 \\
	 47 & 36 & 71 & 141 & -72 \\
	 \end{pmatrix}
	 \]
	 
	 Добавим к 4-ой строке 1-ую, домноженную на -47:
	 \[
	 \begin{pmatrix}
	 1 & 2 & 1 & 3 & -4 \\
	 0 & -29 & 12 & 0 & 58 \\
	 0 & -87 & 25 & 0 & 174 \\
	 0 & -58 & 24 & 0 & 116 \\
	 \end{pmatrix}
	 \]
	 
	 Добавим к 3-ой строке 2-ую, домноженную на -3:
	 \[
	 \begin{pmatrix}
	 1 & 2 & 1 & 3 & -4 \\
	 0 & -29 & 12 & 0 & 58 \\
	 0 & 0 & -11 & 0 & 0 \\
	 0 & -58 & 24 & 0 & 116 \\
	 \end{pmatrix}
	 \]
	 
	 Добавим к 4-ой строке 2-ую, домноженную на -2:
	 \[
	 \begin{pmatrix}
	 1 & 2 & 1 & 3 & -4 \\
	 0 & -29 & 12 & 0 & 58 \\
	 0 & 0 & -11 & 0 & 0 \\
	 0 & 0 & 0 & 0 & 0 \\
	 \end{pmatrix}
	 \]
	 
	 Разделим 3-ую строку на -11:
	 \[
	 \begin{pmatrix}
	 1 & 2 & 1 & 3 & -4 \\
	 0 & -29 & 12 & 0 & 58 \\
	 0 & 0 & 1 & 0 & 0 \\
	 0 & 0 & 0 & 0 & 0 \\
	 \end{pmatrix}
	 \]
	 
	 Добавим к 2-ой строке 3-ую, домноженную на -12:
	 \[
	 \begin{pmatrix}
	 1 & 2 & 1 & 3 & -4 \\
	 0 & -29 & 0 & 0 & 58 \\
	 0 & 0 & 1 & 0 & 0 \\
	 0 & 0 & 0 & 0 & 0 \\
	 \end{pmatrix}
	 \]
	 
	 Добавим к 1-ой строке 3-ую, домноженную на -1:
	 \[
	 \begin{pmatrix}
	 1 & 2 & 0 & 3 & -4 \\
	 0 & -29 & 0 & 0 & 58 \\
	 0 & 0 & 1 & 0 & 0 \\
	 0 & 0 & 0 & 0 & 0 \\
	 \end{pmatrix}
	 \]
	 
	 Разделим 2-ую строку на 29:
	 \[
	 \begin{pmatrix}
	 1 & 2 & 0 & 3 & -4 \\
	 0 & -1 & 0 & 0 & 2 \\
	 0 & 0 & 1 & 0 & 0 \\
	 0 & 0 & 0 & 0 & 0 \\
	 \end{pmatrix}
	 \]
	 
	 Добавим к 1-ой строке 2-ую, домноженную на 2:
	 \[
	 \begin{pmatrix}
	 1 & 0 & 0 & 3 & 0 \\
	 0 & -1 & 0 & 0 & 2 \\
	 0 & 0 & 1 & 0 & 0 \\
	 0 & 0 & 0 & 0 & 0 \\
	 \end{pmatrix}
	 \]
	 
	 $rk\ A = 3$.\\\\\\
	 
	 \subsection{Задача 626}
	 $rk A = r_a$ \\
	 $rk B = r_b$ \\
	 $rk C = r_c$ \\
	 При этом $_a + r_b < r_c$, но любая из строк $C$ выражается через базисные строки $A$ и $B$, то есть $C$ линейно-зависима и ее ранг больше. Тогда по основной лемме о линейной зависимости выходит, что она линейно-зависима, а значит ранг $r_c$ должен быть не большое, чем $r_a + r_b$.
	 
	 \subsection{Задача 627}
	 Следует из предыдущей задачи.
	
	\end{document}