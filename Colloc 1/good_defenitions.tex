\documentclass[a4paper,12pt]{article}
\usepackage{amsmath}
\usepackage{cmap}					% поиск в PDF
\usepackage{mathtext} 				% русские буквы в формулах
\usepackage[T2A]{fontenc}			% кодировка
\usepackage[utf8]{inputenc}			% кодировка исходного текста
\usepackage[english,russian]{babel}	% локализация и переносы
\usepackage{amssymb}				    % Для красивого (!) \mathbb с  буквами и цифрами
\usepackage{mathbbol}

% Изменим формат \section и \subsection:
\usepackage{titlesec}
\titleformat{\section}
{\vspace{1cm}\centering\LARGE\bfseries}	% Стиль заголовка
{}										% префикс
{0pt}									% Расстояние между префиксом и заголовком
{} 										% Как отображается префикс
\titleformat{\subsection}				% Аналогично для \subsection
{\Large\bfseries}
{}
{0pt}
{}

%% Отступы между абзацами и в начале абзаца 
\setlength{\parindent}{0pt}
\setlength{\parskip}{\medskipamount}

\title{Линейная Алгебра и Геометрия\\Определения}
\begin{document}
\maketitle

\begin{enumerate}
\item \textbf{Совместные и несовместные СЛУ.} 

Совместная СЛУ -- это СЛУ, которая имеет хотя бы одно решение. \\
Несовместная СЛУ -- не имеет решений. 

\item \textbf{Эквивалентные СЛУ.} 

Это СЛУ, которые имеют одинакавое множества решений.

\item \textbf{Расширенная матрица системы линейных уравнений.} 
 
Матрица вида $(A | \vec b)$. Где $A$ -- матрица коэффициентов, $\vec b$ -- вектор свободных членов.

\item \textbf{Элементарные преобразования матриц.}  --

Это такие преобразования в результате которых не меняется эквивалентность матриц, т.е. множество решений СЛУ, которому соответствует данная матрица. 
\begin{enumerate}
	\item Прибавление к одной строке матрицы другой строки, домноженной на некоторый коэффициент.
	\item Перестановка двух строк матрицы.
	\item Умножение строки матрицы на некоторый коэффициент.
\end{enumerate}

\item \textbf{Ступенчатый вид матрицы.}
\begin{enumerate}
	\item Номера ведущих элементов(первый ненулевой) строк строго возрастают.
	\item Все нулевые строки стоят в конце.
\end{enumerate}

\item \textbf{Улучшенный ступенчатый вид матрицы.}

\begin{enumerate}
	\item Имеет ступенчатый вид.
	\item Все ведущие элементы строк равны 1 и во всех столбцах, содержащих ведущие элементы, все остальные элементы равны нулю
\end{enumerate}

\item \textbf{Теорема о каноническом виде, к которому можно привести матрицу при помощи элементарных преобразований строк.}

\end{enumerate}

\end{document}
