\documentclass[a4paper,12pt]{article}
\usepackage{amsmath}
\usepackage{cmap}					% поиск в PDF
\usepackage{mathtext} 				% русские буквы в формулах
\usepackage[T2A]{fontenc}			% кодировка
\usepackage[utf8]{inputenc}			% кодировка исходного текста
\usepackage[english,russian]{babel}	% локализация и переносы
\usepackage{amssymb}				    % Для красивого (!) \mathbb с  буквами и цифрами
\usepackage{mathbbol}

% Изменим формат \section и \subsection:
\usepackage{titlesec}
\titleformat{\section}
{\vspace{1cm}\centering\LARGE\bfseries}	% Стиль заголовка
{}										% префикс
{0pt}									% Расстояние между префиксом и заголовком
{} 										% Как отображается префикс
\titleformat{\subsection}				% Аналогично для \subsection
{\Large\bfseries}
{}
{0pt}
{}

%% Отступы между абзацами и в начале абзаца 
\setlength{\parindent}{0pt}
\setlength{\parskip}{\medskipamount}
\begin{document}
	\section{Индивидульное домашнее задание 1 \\ Шумилкин Андрей. Группа 163 \\ Вариант 34. } 
	\subsection{Задача 1} 
	{\sl Программу, которую использовал при решении данной задачи прикрепил к письму с названием class Matrix solve.cpp. Она не полностью автоматическая, то есть очередное преобразование матрицы надо вводить вручную. Решает функция do\_elementary\_operations(). Ввод для этой программы к каждой задаче прикрепил к письму в файлах "*номер задачи* in". Выводит программа ответ сразу в формате кода latex. \\ Так же эта программа решает и третью задачу с помощью реализованных в классе арифметических операцй для матриц. В функции main записан код, который решает третью задачу}.
	
	Запишем расширенную матрицу системы и попытаемся привести ее к каноническому виду:
	
	\[
	\begin{pmatrix}
	78 & 81 & 153 & -318 & 240 \\
	18 & 96 & -42 & -228 & 210 \\
	-40 & -39 & -81 & 158 & -118 \\
	-62 & 66 & -252 & -8 & 70 \\
	\end{pmatrix}
	\]
	
	Разделим 2-ую строку на 2:
	
	\[
	\begin{pmatrix}
	78 & 81 & 153 & -318 & 240 \\
	9 & 48 & -21 & -114 & 105 \\
	-40 & -39 & -81 & 158 & -118 \\
	-62 & 66 & -252 & -8 & 70 \\
	\end{pmatrix}
	\]
	
	Разделим 1-ую строку на 3:
	
	\[
	\begin{pmatrix}
	26 & 27 & 51 & -106 & 80 \\
	9 & 48 & -21 & -114 & 105 \\
	-40 & -39 & -81 & 158 & -118 \\
	-62 & 66 & -252 & -8 & 70 \\
	\end{pmatrix}
	\]
	
	Добавим к 1-ой строке 2-ую, домноженную на -3:
	
	\[
	\begin{pmatrix}
	-1 & -117 & 114 & 236 & -235 \\
	9 & 48 & -21 & -114 & 105 \\
	-40 & -39 & -81 & 158 & -118 \\
	-62 & 66 & -252 & -8 & 70 \\
	\end{pmatrix}
	\]
	
	Домножим 1-ую строку на -1:
	
	\[
	\begin{pmatrix}
	1 & 117 & -114 & -236 & 235 \\
	9 & 48 & -21 & -114 & 105 \\
	-40 & -39 & -81 & 158 & -118 \\
	-62 & 66 & -252 & -8 & 70 \\
	\end{pmatrix}
	\]
	
	Добавим к 2-ой строке 1-ую, домноженную на -9:
	
	\[
	\begin{pmatrix}
	1 & 117 & -114 & -236 & 235 \\
	0 & -1005 & 1005 & 2010 & -2010 \\
	-40 & -39 & -81 & 158 & -118 \\
	-62 & 66 & -252 & -8 & 70 \\
	\end{pmatrix}
	\]
	
	Добавим к 3-ой строке 1-ую, домноженную на 40:
	
	\[
	\begin{pmatrix}
	1 & 117 & -114 & -236 & 235 \\
	0 & -1005 & 1005 & 2010 & -2010 \\
	0 & 4641 & -4641 & -9282 & 9282 \\
	-62 & 66 & -252 & -8 & 70 \\
	\end{pmatrix}
	\]
	
	Добавим к 4-ой строке 1-ую, домноженную на 62:
	
	\[
	\begin{pmatrix}
	1 & 117 & -114 & -236 & 235 \\
	0 & -1005 & 1005 & 2010 & -2010 \\
	0 & 4641 & -4641 & -9282 & 9282 \\
	0 & 7320 & -7320 & -14640 & 14640 \\
	\end{pmatrix}
	\]
	
	Разделим 2-ую строку на -1005:
	
	\[
	\begin{pmatrix}
	1 & 117 & -114 & -236 & 235 \\
	0 & 1 & -1 & -2 & 2 \\
	0 & 4641 & -4641 & -9282 & 9282 \\
	0 & 7320 & -7320 & -14640 & 14640 \\
	\end{pmatrix}
	\]
	
	Добавим к 3-ой строке 2-ую, домноженную на -4641:
	
	\[
	\begin{pmatrix}
	1 & 117 & -114 & -236 & 235 \\
	0 & 1 & -1 & -2 & 2 \\
	0 & 0 & 0 & 0 & 0 \\
	0 & 7320 & -7320 & -14640 & 14640 \\
	\end{pmatrix}
	\]
	
	Добавим к 4-ой строке 2-ую, домноженную на -7320:
	
	\[
	\begin{pmatrix}
	1 & 117 & -114 & -236 & 235 \\
	0 & 1 & -1 & -2 & 2 \\
	0 & 0 & 0 & 0 & 0 \\
	0 & 0 & 0 & 0 & 0 \\
	\end{pmatrix}
	\]
	
	Добавим к 1-ой строке 2-ую, домноженную на -117:
	
	\[
	\begin{pmatrix}
	1 & 0 & 3 & -2 & 1 \\
	0 & 1 & -1 & -2 & 2 \\
	0 & 0 & 0 & 0 & 0 \\
	0 & 0 & 0 & 0 & 0 \\
	\end{pmatrix}
	\]
	
	У нас получилось две ненулевых строки. 
	Запишем получившиеся уравнения:
	
	\[
		\begin{cases}
		x_1 + 3x_3 - 2x_4 = 1 \\
		x_2 - x_3 - 2x_4 = 2
		\end{cases}
	\]
	
	\[
	\begin{cases}
		x_1 = -3x_3 + 2x_4 + 1 \\
		x_2 = x_3 + 2x_4 +2
	\end{cases}
	\]
	
	Тогда общее решение системы линейных уравнений примет вид:
	
	\[
	\begin{pmatrix}
		-3x_3 + 2x_4 + 1 \\
		x_3 + 2x_4 +2 \\
		x_3 \\
		x_4 \\
	\end{pmatrix}
	\]
	
	Найдем одно частное решение. К примеру, возьмем $x_3 = 1$ и $x_4 = -1$:
	
	\[
		\begin{pmatrix}
		-3 - 2 + 1 = -4 \\
		3 - 2 + 2 = 3 \\
		1 \\
		-1 \\
		\end{pmatrix}
	\]
		
	\[
		\begin{pmatrix}
		-4 \\
		3 \\
		1 \\
		-1 \\
	\end{pmatrix}
	\]
	
	\subsection{Задача 2} 
	Запишем расширенную матрицу системы и попытаемся привести ее к каноническому виду:
	
	\[
	\begin{pmatrix}
	32 & 54 & 13 & 112 & 607 \\
	21 & 18 & 20 & 79 & 415 \\
	-3 & 16 & 59 & 131 & 702 \\
	-5 & 2 & 12 & 21 & 143 \\
	\end{pmatrix}
	\]
	
	Поменяем местами 1-ую и 2-ую строку:
	
	\[
	\begin{pmatrix}
	21 & 18 & 20 & 79 & 415 \\
	32 & 54 & 13 & 112 & 607 \\
	-3 & 16 & 59 & 131 & 702 \\
	-5 & 2 & 12 & 21 & 143 \\
	\end{pmatrix}
	\]
	
	Добавим к 1-ой строке 4-ую, домноженную на 4:
	
	\[
	\begin{pmatrix}
	1 & 26 & 68 & 163 & 987 \\
	32 & 54 & 13 & 112 & 607 \\
	-3 & 16 & 59 & 131 & 702 \\
	-5 & 2 & 12 & 21 & 143 \\
	\end{pmatrix}
	\]
	
	Добавим к 2-ой строке 1-ую, домноженную на -32:
	
	\[
	\begin{pmatrix}
	1 & 26 & 68 & 163 & 987 \\
	0 & -778 & -2163 & -5104 & -30977 \\
	-3 & 16 & 59 & 131 & 702 \\
	-5 & 2 & 12 & 21 & 143 \\
	\end{pmatrix}
	\]
	
	Добавим к 3-ой строке 1-ую, домноженную на 3:
	
	\[
	\begin{pmatrix}
	1 & 26 & 68 & 163 & 987 \\
	0 & -778 & -2163 & -5104 & -30977 \\
	0 & 94 & 263 & 620 & 3663 \\
	-5 & 2 & 12 & 21 & 143 \\
	\end{pmatrix}
	\]
	
	Добавим к 4-ой строке 1-ую, домноженную на 5:
	
	\[
	\begin{pmatrix}
	1 & 26 & 68 & 163 & 987 \\
	0 & -778 & -2163 & -5104 & -30977 \\
	0 & 94 & 263 & 620 & 3663 \\
	0 & 132 & 352 & 836 & 5078 \\
	\end{pmatrix}
	\]
	
	Добавим к 2-ой строке 3-ую, домноженную на 8:
	
	\[
	\begin{pmatrix}
	1 & 26 & 68 & 163 & 987 \\
	0 & -26 & -59 & -144 & -1673 \\
	0 & 94 & 263 & 620 & 3663 \\
	0 & 132 & 352 & 836 & 5078 \\
	\end{pmatrix}
	\]
	
	Добавим к 3-ой строке 2-ую, домноженную на 3:
	
	\[
	\begin{pmatrix}
	1 & 26 & 68 & 163 & 987 \\
	0 & -26 & -59 & -144 & -1673 \\
	0 & 16 & 86 & 188 & -1356 \\
	0 & 132 & 352 & 836 & 5078 \\
	\end{pmatrix}
	\]
	
	Разделим 3-ую строку на 2:
	
	\[
	\begin{pmatrix}
	1 & 26 & 68 & 163 & 987 \\
	0 & -26 & -59 & -144 & -1673 \\
	0 & 8 & 43 & 94 & -678 \\
	0 & 132 & 352 & 836 & 5078 \\
	\end{pmatrix}
	\]
	
	Добавим к 2-ой строке 3-ую, домноженную на 3:
	
	\[
	\begin{pmatrix}
	1 & 26 & 68 & 163 & 987 \\
	0 & -2 & 70 & 138 & -3707 \\
	0 & 8 & 43 & 94 & -678 \\
	0 & 132 & 352 & 836 & 5078 \\
	\end{pmatrix}
	\]
	
	Добавим к 3-ой строке 2-ую, домноженную на 4:
	
	\[
	\begin{pmatrix}
	1 & 26 & 68 & 163 & 987 \\
	0 & -2 & 70 & 138 & -3707 \\
	0 & 0 & 323 & 646 & -15506 \\
	0 & 132 & 352 & 836 & 5078 \\
	\end{pmatrix}
	\]
	
	Добавим к 4-ой строке 2-ую, домноженную на 66:
	
	\[
	\begin{pmatrix}
	1 & 26 & 68 & 163 & 987 \\
	0 & -2 & 70 & 138 & -3707 \\
	0 & 0 & 323 & 646 & -15506 \\
	0 & 0 & 4972 & 9944 & -239584 \\
	\end{pmatrix}
	\]
	
	Добавим к 4-ой строке 3-ую, домноженную на -15:
	
	\[
	\begin{pmatrix}
	1 & 26 & 68 & 163 & 987 \\
	0 & -2 & 70 & 138 & -3707 \\
	0 & 0 & 323 & 646 & -15506 \\
	0 & 0 & 127 & 254 & -6994 \\
	\end{pmatrix}
	\]
	
	Добавим к 3-ой строке 4-ую, домноженную на -2:
	
	\[
	\begin{pmatrix}
	1 & 26 & 68 & 163 & 987 \\
	0 & -2 & 70 & 138 & -3707 \\
	0 & 0 & 69 & 138 & -1518 \\
	0 & 0 & 127 & 254 & -6994 \\
	\end{pmatrix}
	\]
	
	Добавим к 4-ой строке 3-ую, домноженную на -1:
	
	\[
	\begin{pmatrix}
	1 & 26 & 68 & 163 & 987 \\
	0 & -2 & 70 & 138 & -3707 \\
	0 & 0 & 69 & 138 & -1518 \\
	0 & 0 & 58 & 116 & -5476 \\
	\end{pmatrix}
	\]
	
	Добавим к 3-ой строке 4-ую, домноженную на -1:
	
	\[
	\begin{pmatrix}
	1 & 26 & 68 & 163 & 987 \\
	0 & -2 & 70 & 138 & -3707 \\
	0 & 0 & 11 & 22 & 3958 \\
	0 & 0 & 58 & 116 & -5476 \\
	\end{pmatrix}
	\]
	
	Добавим к 4-ой строке 3-ую, домноженную на -5:
	
	\[
	\begin{pmatrix}
	1 & 26 & 68 & 163 & 987 \\
	0 & -2 & 70 & 138 & -3707 \\
	0 & 0 & 11 & 22 & 3958 \\
	0 & 0 & 3 & 6 & -25266 \\
	\end{pmatrix}
	\]
	
	Добавим к 3-ой строке 4-ую, домноженную на -4:
	
	\[
	\begin{pmatrix}
	1 & 26 & 68 & 163 & 987 \\
	0 & -2 & 70 & 138 & -3707 \\
	0 & 0 & -1 & -2 & 105022 \\
	0 & 0 & 3 & 6 & -25266 \\
	\end{pmatrix}
	\]
	
	Добавим к 4-ой строке 3-ую, домноженную на 3:
	
	\[
	\begin{pmatrix}
	1 & 26 & 68 & 163 & 987 \\
	0 & -2 & 70 & 138 & -3707 \\
	0 & 0 & -1 & -2 & 105022 \\
	0 & 0 & 0 & 0 & 289800 \\
	\end{pmatrix}
	\]
	
	Данная система уравнений несовместна, т.к. $0 \not= 289800$.
	
	\subsection{Задача 3} 
	
	Примем: \\
	$A = \begin{pmatrix} 7 & 14 \\ 7 & 10 \\ \end{pmatrix}.\\$
	$B = \begin{pmatrix} 4 & 3 \\ 7 & 6 \\ \end{pmatrix}.\\$
	$C = \begin{pmatrix} 4 & 1 \\ 7 & 5 \\ \end{pmatrix}.\\$
	$D = \begin{pmatrix} 3 & 4 \\ 7 & 4 \\ \end{pmatrix}.\\$
	
	Тогда можем переписать данное выражение в виде: \\
	$A \cdot B \cdot C - \biggl[ C \cdot A^T \biggr]^T \cdot C + A \cdot D \cdot C - B^T \cdot A^T \cdot A + A \cdot \biggl[ A^T \cdot C \biggr]^T - D^T \cdot A^T \cdot A$ = 
	$A \cdot B \cdot C - A \cdot C^T \cdot C + A \cdot D \cdot C - B^T \cdot A^T \cdot A + A \cdot C^T \cdot A - D^T \cdot A^T \cdot A.$ = 
	$A \cdot (B  - C^T  + D) \cdot C + (-B^T \cdot A^T + A \cdot C^T - D^T \cdot A^T) \cdot A$
	
	Это равно:
	
	\[
	\begin{pmatrix}
	-609 & -2989 \\
	-453 & -2213 \\
	\end{pmatrix}
	\]
	
	\subsection{Задача 4} 
	
	\[
	\begin{pmatrix}
	3 & 0 & 1 \\
	0 & 9 & 1 \\
	0 & 0 & 3 \\ 
	\end{pmatrix} = A
	\]
	
	\[
	\begin{pmatrix}
	a_{11} & a_{12}  & a_{13}  \\
	a_{21}  & a_{22}  & a_{23}  \\
	a_{31}  & a_{32}  & a_{33}  \\
	\end{pmatrix}
	\]			
	
	Заметим, что четыре элемента в любой степени будут равны нулю. Обозначим следующую степень элемента как $a^{n+1}$: 
	
	$a_{12}^{n+1} = a_{11}^{n} \cdot a_{12} + a_{12}^{n} \cdot a_{22} + a_{13}^{n} \cdot a_{32} = a_{11}^{n} \cdot 0 + 0 \cdot a_{22} + a_{13}^{n} \cdot 0 = 0$ \\ 
	$a_{32}^{n+1} = a_{31}^{n} \cdot a_{12} + a_{32}^{n} \cdot a_{22} + a_{33}^{n} \cdot a_{32} = 0 \cdot a_{12} + 0 \cdot a_{22} + a_{33}^{n} \cdot 0 = 0.$\\
	$a_{21}^{n+1} = a_{21}^{n} \cdot a_{11} + a_{22}^{n} \cdot a_{21} + a_{23}^{n} \cdot a_{31} = 0 \cdot a_{11} + a_{22}^{n} \cdot 0 + a_{23}^{n} \cdot 0 = 0.$ \\
	$a_{31}^{n+1} = a_{31}^{n} \cdot a_{11} + a_{32}^{n} \cdot a_{21} + a_{33}^{n} \cdot a_{31} = 0 \cdot a_{11} + 0 \cdot a_{21} + a_{33}^{n} \cdot 0 = 0.$ \\
	
	Тогда: \\
	$a_{11}^{n+1} = a_{11}^{n} \cdot a_{11} + a_{12}^{n} \cdot a_{21} + a_{13}^{n} \cdot a_{31} = a_{11}^{n} \cdot 3 + 0 \cdot 0 + a_{13}^{n} \cdot 0 =  3 \cdot a_{11}^{n}$.
	$a_{22}^{n+1} = a_{21}^{n} \cdot a_{12} + a_{22}^{n} \cdot a_{22} + a_{23}^{n} \cdot a_{32} = 0 \cdot 0 + a_{22}^{n} \cdot a_{22} + a_{23}^{n} \cdot 0 =  9 \cdot a_{22}^{n}$.
	$a_{33}^{n+1} = a_{31}^{n} \cdot a_{13} + a_{32}^{n} \cdot a_{23} + a_{33}^{n} \cdot a_{33} = 0 \cdot a_{13} + 0 \cdot a_{23} + a_{33}^{n} \cdot a_{33} = 3 \cdot a_{33}^{n}$.
	
	И наконец: \\
	$a_{13}^{n+1} = a_{11}^{n} \cdot a_{13} + a_{12}^{n} \cdot a_{23} + a_{13}^{n} \cdot a_{33} = a_{11}^{n} \cdot 1 + 0 \cdot 1 + a_{13}^{n} \cdot 3 = a_{11}^{n} + 3 \cdot a_{13}^{n}.$ \\
	$a_{23}^{n+1} = a_{21}^{n} \cdot a_{13} + a_{22}^{n} \cdot a_{23} + a_{23}^{n} \cdot a_{33} = 0 \cdot a_{13} + a_{22}^{n} \cdot 1 + a_{23}^{n} \cdot 3 = a_{22}^{n} + 3 \cdot a_{23}^{n}.$
	
	В итоге имеем: \\
	$a_{12}^{n} = 0.$ \\
	$a_{21}^{n} = 0.$ \\	
	$a_{31}^{n} = 0.$ \\
	$a_{32}^{n} = 0.$ \\
	$a_{11}^{n} = 3^n.$ \\
	$a_{22}^{n} = 9^{n}.$ \\
	$a_{33}^{n} = 3^{n}.$ \\
	$a_{13}^{n} = n \cdot 3^{n-1}.$ \\
	$a_{23}^{n} = \frac{3^{n-1} \cdot (3^n-1)}{2}.$ \\
	
	\[
	\begin{pmatrix}
	3^{n} & 0  & n \cdot 3^{n-1}  \\
	0  & 9^{n}  &  	\frac{3^{n-1} \cdot (3^n-1)}{2}  \\
	0  & 0  & 3^{n} \\
	\end{pmatrix}
	\]		
	
	
	\end{document}